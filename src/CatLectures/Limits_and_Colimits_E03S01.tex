\documentclass{article}

\newcommand{\bra}[1]{\left(#1\right)}
\usepackage[activate={true,nocompatibility},final,tracking=true,kerning=true,spacing=true,factor=1100,stretch=10,shrink=10]{microtype}
\microtypecontext{spacing=nonfrench}
\usepackage{tikz}
\usepackage{tikz-cd}
\usepackage{mathpazo}
\usepackage{stmaryrd}
\usepackage{amsmath,amsthm,amssymb}
\usepackage{subcaption}
\usepackage{enumerate}
\usetikzlibrary{shapes}
\usetikzlibrary{positioning}
\usetikzlibrary{decorations.pathmorphing}
% Set up the images/graphics package
\usepackage{graphicx,float}
\setkeys{Gin}{width=\linewidth,totalheight=\textheight,keepaspectratio}
\graphicspath{{.}}


% Small sections of multiple columns
\usepackage{multicol}
\usepackage[margin=1.6in]{geometry}

%--------Theorem Environments--------
%theoremstyle{plain} --- default
\newtheorem{thm}{Theorem}
\newtheorem{cor}[thm]{Corollary}
\newtheorem{prop}[thm]{Proposition}
\newtheorem{lem}[thm]{Lemma}
\newtheorem{fact}[thm]{Fact}
\newtheorem{conj}[thm]{Conjecture}
\newtheorem{quest}[thm]{Question}
\newtheorem{claim}{Claim}

\theoremstyle{definition}
\newtheorem{defn}[thm]{Definition}
\newtheorem{defns}[thm]{Definitions}
\newtheorem{con}[thm]{Construction}
\newtheorem{exmp}[thm]{Example}
\newtheorem{jk}[thm]{Joke}
\newtheorem{exmps}[thm]{Examples}
\newtheorem{notn}[thm]{Notation}
\newtheorem{notns}[thm]{Notations}
\newtheorem{addm}[thm]{Addendum}
\newtheorem{exer}[thm]{Exercise}

\theoremstyle{remark}
\newtheorem{rem}[thm]{Remark}
\newtheorem{ans}[thm]{Answer}
\newtheorem{rems}[thm]{Remarks}
\newtheorem{warn}[thm]{Warning}
\newtheorem{sch}[thm]{Scholium}

% MACROS
\newcommand{\Mod}[1]{\ (\text{mod}\ #1)}
\newcommand{\R}{\mathbb{R}}
\newcommand{\N}{\mathbb{N}}
\newcommand{\Q}{\mathbb{Q}}
\newcommand{\F}{\mathbb{F}}
\newcommand{\Z}{\mathbb{Z}}
\newcommand{\mC}{\mathcal{C}}
\newcommand{\mG}{\mathcal{G}}
\newcommand{\mP}{\mathcal{P}}
\newcommand{\one}{\mathbb{1}}
\renewcommand{\P}{\mathbb{P}}
\DeclareMathOperator{\dist}{dist}
\DeclareMathOperator{\aut}{Aut}
\DeclareMathOperator{\gal}{Gal}
\DeclareMathOperator{\orb}{Orb}
\DeclareMathOperator{\stab}{Stab}
\DeclareMathOperator{\inn}{Inn}
\DeclareMathOperator{\spn}{Span}
\DeclareMathOperator{\out}{Out}
\DeclareMathOperator{\im}{Im}
\DeclareMathOperator{\arr}{Arr}
\DeclareMathOperator{\rk}{rk}
\DeclareMathOperator{\rcf}{rcf}
\DeclareMathOperator{\tors}{Tors}
\DeclareMathOperator{\Hom}{Hom}
\DeclareMathOperator{\ann}{Ann}
\DeclareMathOperator{\syl}{Syl}
\DeclareMathOperator*{\colim}{co{\lim}}
\newcommand{\norm}[1]{\left\lVert #1 \right\rVert}
\newcommand{\inp}[2]{\left\langle #1, #2 \right\rangle}
\newcommand{\id}{\text{id}}
\newcommand{\gln}{\text{GL}_n}
\newcommand{\op}[1]{#1^{\text{op}}}
\newcommand{\pullback}{\mbox{\LARGE$\lrcorner$}}
\newcommand{\pushout}{\mbox{\LARGE$\ulcorner$}}


\title{Lecture 3 - Limits and Colimits\vspace{-10pt}}
\date{\vspace{-30pt}\today\vspace{-15pt}}  % if the \date{} command is left out, the current date will be used
\begin{document}
\maketitle
\begin{abstract} Some simple constructions in \textbf{Set} are given in categorical terms in order to introduce limits and colimits, the general formalization is briefly discussed.
\end{abstract}
The unifying power of categorical abstraction is arguably its biggest benefit. Indeed, it is often the case that many mathematical objects or results from different fields fit under the same categorical definition or fact. In the author's opinion, category is at its peak of elegance when a complex idea becomes trivial when viewed categorically, and when this same view helps link together the intuitions behind many ideas throughout mathematics. 

The next two lectures concern one particular instance of this power, that is, the use of \textbf{universal properties} to define mathematical constructions. This term is somewhat delicate to define, therefore, we postpone its definition to next lecture and for a while, we suggest the reader to try and recognize \textit{universality} as the thing that all definitions of (co)limits have in common. This first lecture will cover \textbf{limits and colimits} which are specific cases of universal constructions.

The first section presents several examples; each of its subsection is dedicated to one kind of limit or colimit of which a detailed example in \textbf{Set} is given along with a couple of interesting examples. The second section gives a formal framework to talk about all the examples previously explored as well as a few general results. In the sequel, $\mathbf{C}$ denotes a category.

\section{Examples}
\subsection{Product}
Given two sets $S$ and $T$, the most common construction of $S \times T$ is conceptually easy, you take all pairs of elements $S$ and $T$, that is,
\[S\times T := \left\{ (s,t) \mid s \in S, t \in T\right\}.\]
However, this does not have a nice categorical analog because it requires to pick out elements in $S$ and $T$. If one hopes to generalize products to other categories, the construction must only involve objects and morphisms. What are significant functions (morphisms) to consider when studying $S \times T$? Projection maps.

They are functions $\pi_1: S\times T \rightarrow S$ and $\pi_2: S\times T \rightarrow T$, but that is not enough to define the product. Indeed, the set $S\times T \times S$ has the same projection maps while it is clearly not always isomorphic to $S\times T$. What is special about $S\times T$ with the projections? For one, $\pi_1$ and $\pi_2$ are surjective and while they are not injective, they have an invertible-like property. Namely, given $s \in S$ and $t \in T$, the pair $(s, t)$ is completely determined from $\pi_1^{-1}(s) \cap \pi_2^{-1}(t)$.

Again, in order to discharge the references to specific elements, another point of view is needed. Let $X$ be a set of \textit{choices} of pairs, an element $x \in X$ chooses elements in $S$ and $T$ via functions $c_1 : X \rightarrow S$ and $c_2: X \rightarrow T$ (similar to the projections). Now, the ``inverse'' defined above yields a function 
\[!:X \rightarrow S\times T = x \mapsto \pi^{-1}(c_1(x)) \cap \pi^{-1}(c_2(x)).\]
This function maps $x \in X$ to an element in $S\times T$ that makes the same choice and it is the only one that does so. Categorically, $!$ is the unique morphism in $\Hom_C(X, S\times T)$ satisfying $\pi_i\circ ! = c_i$ for $i =1,2$. Later, we will see that this property completely determines $S\times T$. For now, enjoy the power we gain from generalizing this idea.
\begin{defn}[Binary product]
    Let $A, B \in \mathbf{C}_0$. A (categorical) \textbf{binary product} of $A$ and $B$ is an object, denoted $A \times B$, along with two morphisms $\pi_A: A \times B \rightarrow A$ and $\pi_B: A \times B \rightarrow B$ called \textbf{projections} that satisfy the following \textbf{universal property}: for every object $X \in \mathbf{C}_0$ with morphisms $f_A: X\rightarrow A$ and $f_B:X \rightarrow B$, there is a unique morphism $!: X \rightarrow A \times B$ making diagram \eqref{diag-produniv} commute.
    \begin{equation}
    \begin{tikzcd}\label{diag-produniv}
        & X \arrow[ld, "f_A"'] \arrow[rd, "f_B"] \arrow[d, "!", dashed] &   \\
        A & A\times B \arrow[l, "\pi_A"] \arrow[r, "\pi_B"']              & B
    \end{tikzcd}
    \end{equation}
    It is common to denote $! = (f_A, f_B)$.
\end{defn}
\begin{exmp}[\textbf{Set}]
    Cleaning up the argument above, we show that the cartesian product $A \times B$ with the usual projections is a categorical product in \textbf{Set}. To show that it satisfies the universal property, let $X$, $f_A$ and $f_B$ be as in the definition. A function $!:X\rightarrow A \times B$ that makes \eqref{diag-produniv} commute must satisfy
    \[\forall x, \pi_A(!(x)) = f_A(x) \text{ and } \pi_B(!(x)) = f_B(x).\]
    Equivalently, $!(x) = (f_A(x), f_B(x))$. Since this uniquely determines $!$, $A \times B$ is indeed the categorical product.
\end{exmp}
\begin{exmps}
    Most of the constructions throughout mathematics with the name product can also be realized with a categorical product. Examples include the direct product of groups, rings or vector spaces, the product of topologies, and so on\dots The fact that all these constructions are based on the cartesian product of the underlying sets is a corollary of a deeper result about the forgetful functor that all these categories have in common (cf. Lecture 6).

%T: je suis pas fan de noter le produits des ouverts U et V UxV, autant lui donner un autre nom (W par exemple)?
    In another flavour, let $X$ be a topological space and $\mathcal{O}(X)$ be the category of opens. If $A, B \subseteq X$ are open, what is their product? Following the definition, the existence of $\pi_A$ and $\pi_B$ imply that $A\times B$\footnote{Recall that $\times$ denotes the categorical product, not the cartesian product of sets.} is included in both sets, or equivalently $A \times B \subseteq A \cap B$.
    
    Moreover, for any open set $X$ included in $A$ and $B$ (via $f_A$ and $f_B$), $X$ should be included in $A \times B$ (via $!$, uniqueness is already given in a posetal category). In particular, $X$ can be $A \cap B$ (it is open by definition of a topology), thus $A \cap B \subseteq A \times B$. In conclusion, the product of two open sets is their intersection. In a general poset, the same argument is used to show the product is the greatest lower bound/infimum/meet.
\end{exmps}
\begin{rem}
    Given two objects in an arbitrary category, their product does not necessarily exist. Nevertheless, when it exists, one can (and we will) show that it is unique up to unique isomorphism. Thus, in the sequel, we will speak of \textit{the} product of two objects and similarly for other constructions presented in this lecture.
\end{rem}
To generalize the categorical product to more than two objects, one can, for instance, define the product of a finite family of sets recursively with the binary product. However, this implies having to show the associativity and commutativity of $\times$ (heavily uses uniqueness, cf. Exercises) for it to be well-defined. In contrast, generalizing the universal property illustrated in \eqref{diag-produniv} yields a simpler definition that works even for arbitrary families.
\begin{rem}
    In the case of the category of open subsets of a topological space, the arbitrary product is not always the intersection. This is because arbitrary intersections of open sets are not necessarily open. To resolve this problem, it suffices to take the interior of the intersection.
\end{rem}
\begin{defn}[Product]
    Let $\{X\}_{i \in I}$ be an $I$-indexed family of objects of $\mathbf{C}$. The \textbf{product} of this family is an object, denoted $\prod_{i \in I} X_i$ along with projections $\pi_j: \prod_{i \in I} X_i \rightarrow X_j$ for all $j \in I$ satisfying the following universal property: for any object $X$ with morphisms $\left\{ f_j: X\rightarrow X_j\right\}_{j \in I}$, there is a unique morphism $!: X \rightarrow \prod_{i \in I} X_i$ making \eqref{diag-arbitraryproduniv} commute for all $j \in I$.
    \begin{equation}\label{diag-arbitraryproduniv}
        \begin{tikzcd}
        X \arrow[d, "!"', dashed] \arrow[rd, "f_j"] &     \\
        \prod_{i \in I}X_i \arrow[r, "\pi_j"']     & X_j
        \end{tikzcd}
    \end{equation}
\end{defn}
\begin{rem}
    A family of objects in a category is also called a \textbf{discrete diagram/system}, the product is then the \textbf{limit} of this diagram.
\end{rem}
The big takeaway from last lecture is that the natural thing to do after reading this definition is to dualize it. So we ask, what is the \textbf{colimit} of a discrete diagram?
\subsection{Coproduct}
\begin{defn}[Coproduct]
    Let $\{X\}_{i \in I}$ be an $I$-indexed family of objects in $\mathbf{C}$, its \textbf{coproduct} is an object, denoted $\coprod_{i \in I} X_i$ (or $X_1 + X_2$ in the binary case), along with morphisms $\kappa_j: X_j \rightarrow \coprod_{i \in I} X_i$ for all $j \in I$ called \textbf{coprojections} satisfying the following universal property: for any object $X$ with morphisms $\left\{ f_j: X_j \rightarrow X\right\}_{j\in I}$, there is a unique morphism $!: \coprod_{i \in I}X_i \rightarrow X$ making \eqref{diag-arbitrarycoproduniv} commute for all $j \in I$.
    \begin{equation}\label{diag-arbitrarycoproduniv}
        \begin{tikzcd}
        X_j  \arrow[r, "\kappa_j"] \arrow[rd, "f_j"'] &\coprod_{i \in I}X_i\arrow[d, "!", dashed]\\
        & X
        \end{tikzcd}
    \end{equation}
    In the finite case, it is common to denote $! = [f_1, \dots, f_n]$. %TODO: explain the notaion $\times$ and $+$ for functions which is different from $()$ and $[]$.
\end{defn}
Surprisingly, this setting leads to more complex examples.
\begin{exmp}[\textbf{Set}]
    Let $\{X_i\}_{i \in I}$ be a family of sets, first note that if $X_j = \emptyset$ for $j \in I$, then there is only one morphism $X_j \rightarrow X$ for any $X$. In particular, \eqref{diag-arbitrarycoproduniv} commutes no matter what $\coprod_{i \in I} X_i$ and $X$ are. Therefore, removing $X_j$ from this family does not change how the coproduct behaves, hence no generality is loss from assuming all $X_i$'s are non-empty.

    Second, for any $j \in I$, let $X = X_j$, $f_j = \id_{X_j}$ and for any $j' \neq j$, let $f_{j'}$ be any morphism in $\Hom(X_{j'}, X_j)$ (one exists because $X_j$ is non-empty). Commutativity of \eqref{diag-arbitrarycoproduniv} implies $\kappa_j$ has a left inverse because $! \circ \kappa_j = f_j = \id_{X_j}$, so all coprojections are injective.

    Third, we claim that for any $j \neq j' \in I$, $\im(\kappa_j) \cap \im(\kappa_{j'}) = \emptyset$. Assume towards a contradiction that there exists $j\neq j' \in I$, $x \in X_j$ and $x' \in X_{j'}$ such that $\kappa_j(x) = \kappa_{j'}(x')$. Then, let $X = \{0,1\}$, $f_j \equiv 0$ and $f_{j'} \equiv 1$ (the other morphisms can be chosen arbitrarily). The universal property implies that $! \circ \kappa_j \equiv 0$ and $!\circ \kappa_{j'} \equiv 1$, but it contradicts $!(\kappa_j(x)) = !(\kappa_{j'}(x'))$.

    Finally, the previous point says that $\coprod_{i\in I} X_i$ contains distinct copies of the images of all coprojections. Furthermore, the $\kappa_j$'s being injective, their image can be identified with the $X_j$'s to obtain \[\bigsqcup_{i \in I} X_i \subseteq \coprod_{i \in I} X_i,\]
    where $\sqcup$ denotes the disjoint union. For the converse inclusion, in \eqref{diag-arbitrarycoproduniv}, let $X$ be the disjoint union and the $f_j$'s be the inclusions. Assume there exists $x$ in the R.H.S. that is not in the L.H.S., then we can define $!': \coprod_{i \in I} X_i\rightarrow \bigsqcup_{i \in I} X_i$ that only differs from $!$ at $x$. Since $x$ is not in the image of any of the $\kappa_j$, the diagrams still commute and this contradicts the uniqueness of $!$.

    In conclusion, the coproduct in \textbf{Set} is the disjoint union (this gives the intuition to why empty sets were not considered).
\end{exmp}
\begin{exmps}
    \textbf{In the category of open sets of} $(X, \tau)$: let $\{U_i\}_{i \in I}$ be a family of open sets and suppose $\coprod_i U_i$ exists. The coprojections yield inclusions $U_j \subseteq \coprod_i U_i$ for all $j \in I$, so $\coprod_i U_i$ must contain all $U_j$'s and thus $\cup_i U_i$. Moreover, in \eqref{diag-arbitrarycoproduniv}, letting $f_j$ be the inclusion $U_j \hookrightarrow \cup_iU_i$ for all $j\in I$  (it lives in this category because $\cup_i U_i$ is open), the existence of $!$ yields an inclusion $\coprod_i U_i \subseteq \cup_i U_i$. We conclude that the coproduct in this category is just the union. In a general poset, the same argument is used to show the coproduct is the least upper bound/supremum/join.
    %T:le produit tensoriel n'est pas un produit au sens catégorique ! 
    \textbf{In} $\textbf{Vsp}_k$: the coproduct, also called the direct sum, is defined by\footnote{Here, the symbol $\prod$ denotes the cartesian product of the $V_i$'s as sets. The categorical product of vector spaces is also the direct sum, where the projections are the usual ones.} 
        \[\coprod_{i \in I} V_i = \{v \in \prod_{i\in I}V_i \mid v(i) \neq 0 \text{ for finitely many $i$'s}\},\]
        where $\kappa_j: V_j \hookrightarrow \coprod_i V_i$ sends $v$ to $\bar{v} \in \prod_i V_i$ with $\bar{v}_j = v$ and $\bar{v}_{j'} = 0$ whenever $j \neq j'$. To verify this, let $\left\{ f_j: V_j \rightarrow X\right\}_{j \in I}$ be a family of linear maps. We can construct $!$ by defining it on basis elements of the direct sum, which are just the basis elements of all $V_j$'s seen as elements of the big sum (via the coprojections). Indeed, if $b$ is in the basis of $V_j$, we let $!(\bar{b}) = f_j(b)$. Extending linearly yields a linear map $!: \coprod_iV_i \rightarrow X$. Uniqueness is clear because if $h:\coprod_iV_i \rightarrow X$ differs from $!$ on one of the basis elements, it does not make \eqref{diag-arbitrarycoproduniv} commute. Notice that it is necessary to require finitely many non-zero entries, otherwise the basis of the coproduct would not be union of all bases of the $V_j$'s.
\end{exmps}

In a very similar way to the product and coproduct, we will define various constructions in \textbf{Set} from the categorical point of view.
\subsection{Equalizer}
\begin{defn}[Fork]
    A \textbf{fork} in $\mathbf{C}$ is a diagram of shape \eqref{diag-fork} or \eqref{diag-cofork} that commutes.\\
    \begin{minipage}{0.47\textwidth}
        \begin{equation}\label{diag-fork}
            \begin{tikzcd}
                O \arrow[r, "o"]                         & A \arrow[r, "f", shift left] \arrow[r, "g"', shift right] & B
            \end{tikzcd}
        \end{equation}
    \end{minipage}
    \begin{minipage}{0.47\textwidth}
        \begin{equation}\label{diag-cofork}
            \begin{tikzcd}
                A \arrow[r, "f", shift left] \arrow[r, "g"', shift right] & B \arrow[r, "o"] & O
            \end{tikzcd}
        \end{equation}
    \end{minipage}
\end{defn}
\begin{defn}[Equalizer]
    Let $A, B \in \mathbf{C}_0$ and $f,g:A\rightarrow B$ be parallel morphisms. The \textbf{equalizer} of $f$ and $g$ is an object $E$ and a morphism $e:E\rightarrow A$ satisfying $f\circ e = g \circ e$ with the following universal property: for any object $O$ with morphism $o:O\rightarrow A$ satisfying $f\circ o = g \circ o$, there is a unique $!: O \rightarrow E$ making \eqref{diag-equalizer} commute.
    \begin{equation}\label{diag-equalizer}
        \begin{tikzcd}
            O \arrow[rd, "o"] \arrow[d, "!"', dashed] &                                                           &   \\
            E \arrow[r, "e"']                         & A \arrow[r, "f", shift left] \arrow[r, "g"', shift right] & B
            \end{tikzcd}
    \end{equation}
\end{defn}
\begin{exmp}[\textbf{Set}]
    Let $f,g:A\rightarrow B$ be two functions and suppose $e:E\rightarrow A$ is their equalizer. By associativity, for any $h:O\rightarrow E$, the composite $e\circ h$ is a candidate for $o$ in diagram \eqref{diag-equalizer} because $f\circ (e \circ h) = g \circ (e \circ h)$. What is more, if $h'$ is such that $e \circ h = e\circ h'$, then $h= h'$ or it would contradict the unicity of $!$. In other words, $e$ is monic/injective.

    This implies $E$ is isomorphic to its image under $e$. By the first property of $e$, its image is contained in the subset $\left\{ a \in A \mid f(a) = g(a)\right\}$. But, by the universal property, letting $O$ be this set and $o$ be the inclusion, there is an injection: $!: \left\{ a \in A \mid f(a) = g(a)\right\} \rightarrow E$, thus both sets are isomorphic. In conclusion, the equalizer of two parallel functions is the subset $E$ in which they are equal and $e:E \hookrightarrow A$ is the inclusion.
\end{exmp}
\begin{exmps}
    \textbf{In a posetal category}: $\Hom$-sets are singletons, so it must be the case that $f = g$ and any $o:O \rightarrow A$ satisfies $f\circ o = g\circ o$. Written using the order notation, the universal property is then equivalent to the fact that $O \leq A$ implies $O \leq E$. In particular, if $O= A$, then $A\leq E$, so $A = E$ by antisymmetry.

    \textbf{In Ab, Ring or $\textbf{Vsp}_k$}: For the same reason that the cartesian product of the underlying sets is the underlying set of the product, the construction of equalizers is as in \textbf{Set}. Nevertheless, since each of these categories have a notion of additive inverse for morphisms, the equalizer of $f$ and $g$ has a cooler name, that is, $\ker(f-g)$.

    %TODO: can you find more examples?
\end{exmps}

\subsection{Coequalizer}

\begin{defn}[Coequalizer]
    Let $A, B \in \mathbf{C}_0$ and $f,g:A\rightarrow B$ be parallel morphisms. The \textbf{coequalizer} of $f$ and $g$ is an object $D$ and a morphism $d:B\rightarrow D$ satisfying $d\circ f = d \circ g$ with the following universal property: for any object $O$ with morphism $o:B\rightarrow O$ satisfying $o\circ f = o \circ g$, there is a unique $!: D \rightarrow O$ making \eqref{diag-coequalizer} commute.
    \begin{equation}\label{diag-coequalizer}
        \begin{tikzcd}
            A \arrow[r, "f", shift left] \arrow[r, "g"', shift right] & B \arrow[r, "d"] \arrow[rd, "o"'] & D \arrow[d, "!", dashed] \\& & O
        \end{tikzcd}
    \end{equation}
\end{defn}
\begin{exmp}[\textbf{Set}]
    Let $f,g:A\rightarrow B$ be two functions an suppose $d:B \rightarrow D$ is their coequalizer. Similarly to the dual case, one can show that $d$ is surjective. Since $d\circ f = d \circ g$, for any $b, b' \in B$,
    \begin{equation}\label{eqn-relation}\tag{$*$}
        \exists a \in A, f(a) = b\text{ and } g(a) = b' \implies d(b) =d(b').
    \end{equation} Denoting $\sim$ to be the relation in the L.H.S. of \eqref{eqn-relation}, the implication is $b \sim b' \implies d(b) = d(b')$. However, note that $\sim$ is not an equivalence relation while $=$ is, thus, the converse implication does not always hold. For instance, when $b\sim b'\sim b''$, $d(b) = d(b'')$, but it might not be the case that $b\sim b''$.

    Consequently, it makes sense to consider the equivalence relation generated by $\sim$ (in this case, it is simply the transitive closure), denoted $\simeq$. As noted above, the forward implication $b\simeq b' \implies d(b)= d(b')$ still holds. For the converse, in \eqref{diag-coequalizer}, let $O:= B/{\simeq}$ and $o: B \rightarrow B/{\simeq}$ be the quotient map, by composing with $!$, we have that \[d(b) = d(b') \implies o(b) = o(b') \implies b \simeq b'.\]
    In conclusion, $D= B/{\simeq}$ and $d:B \rightarrow D$ is the quotient map.
\end{exmp}
\begin{exmps}
    \textbf{In a posetal category}: an argument dual to the one for equalizers shows the coequalizer of $f,g:A \rightarrow B$ is $B$.

    \textbf{In Ab, Ring or $\textbf{Vsp}_k$}: Let $f,g: A \rightarrow B$ be homomorphisms and suppose $d:B \rightarrow D$ is their coequalizers. Consider the function $f-g$, the first property of $d$ implies $d \circ (f-g) = 0$, or equivalently, $\im(f-g) \subseteq \ker(d)$. Now, consider diagram \eqref{diag-coeqinabelian} as a particular instance of \eqref{diag-coequalizer} because $q\circ (f-g) = 0$, where $q$ is the quotient map.
    \begin{equation}\label{diag-coeqinabelian}
        \begin{tikzcd}
            A \arrow[r, "f", shift left] \arrow[r, "g"', shift right] & B \arrow[r, "d"] \arrow[rd, "q"'] & D \arrow[d, "!", dashed] \\& & B/{\im(f-g)}
        \end{tikzcd}
    \end{equation}
    We claim that $!$ has an inverse, implying that $D \cong B/\im(f-g)$. Indeed, for $[x] \in B/\im(f-g)$, we must have 
    \[!^{-1}([x]) = !^{-1}(q(x)) = !^{-1}(!(d(x)))= d(x),\]
    and it is only left to show $!^{-1}$ is well-defined because the inverse of a homomorphism is a homomorphism. This follows because if $[x] = [x']$, then there exists $y \in \im(f-g)$ such that $x = x' + y$, so \[!^{-1}(x) = d(x) = d(x'+y) = d(x') +d(y) = d(x') + 0 = !^{-1}(x').\]
    In the special case that $g \equiv 0$, $B/\im(f)$ is called the \textbf{cokernel} of $f$, denoted $\text{coker}(f)$.

    \textbf{Monoid presentations}: Let $M$ be a monoid, recall that a set $A \subseteq M$ \textbf{generates} $M$, denoted $M = \langle A \rangle$, if any element of $M$ is a finite product of elements of $A$. Namely, for any $m \in M$, there exists $a_1,\dots, a_n \in A$ such that $a_1 \cdots a_n = m$. If we consider the set of all finite products on $A$, call it $F(A)$, $M = \langle A \rangle$ yields a surjection $F(A) \rightarrow M$. However, the converse is not true because such a surjection does not necessarily behave well with the monoid operation.

    However, there is a natural monoid operation on $F(A)$, that is concatenation: \[(a_1\cdots a_n) \cdot (a_1'\cdots a_m') = a_1\cdots a_na_1'\cdots a_m',\]
    with the empty product as the identity (even if $1_M \in A$ because $1_Ma \neq a$ as elements of $F(A)$). Now, a surjective homomorphism $d: F(A) \twoheadrightarrow M$ does imply $M = \langle A \rangle$. Indeed, a product $a_1\cdots a_n$ in the preimage of $m$ has to equal $m$ inside $M$ or it would contradict the homomorphism property.

    By the first isomorphism theorem, $M$ is isomorphic to $F(A)/\ker(d)$. To realize $d$ as a coequalizer, we will find a morphism $f$ such that $\text{coker}(f)$ is $M \cong F(A)/\ker(d)$, namely, we need to find $f: X \rightarrow F(A)$ with $\im(f) = \ker(d)$.\footnote{In this category $g$ is not $0$ but $1$ everywhere.} This is similar to what we were doing at the start of this example. Indeed, let $R \subseteq F(A)$ be a set of generators of $\ker(d)$, then there is a morphism $f: F(R) \rightarrow F(A)$ satisfying $\im(f) = \ker(d)$. In fact, we can take the morphism $f$ that simply views products of products of $A$ as products of $A$ by concatentation. We have shown that \eqref{diag-forkpresentation} forms a fork and the argument used in \textbf{Ab} can be applied here to show this is a coequalizer.
    \begin{equation}\label{diag-forkpresentation}
        \begin{tikzcd}
            F(R) \arrow[r, "f", shift left] \arrow[r, "1"', shift right] & F(A) \arrow[r, two heads, "d"] & M\cong F(A)/\ker(d)
        \end{tikzcd}
    \end{equation}
    Thus, one can see $M$ as generated by $A$ subject to $R$ that identify some products of $A$ with the identity. Elements of $R$ are called \textbf{relations} and the pair $A$ and $R$ are a \textbf{presentation} of $M$, denoted $M = \langle A \mid R \rangle$.
\end{exmps}

\subsection{Pullback}
\begin{defn}[Cospan]
    A \textbf{cospan} in $\mathbf{C}_0$ is comprised of three objects $A,B,C$ and two morphisms $f$ and $g$ as follows. This is simply a shape of diagram that has been given a name as it (and its dual) occurs quite often.
    \begin{equation*}
        \begin{tikzcd}
            A \arrow[r, "f"] & C & B \arrow[l, "g"']
        \end{tikzcd}
    \end{equation*}
\end{defn}
\begin{defn}[Pullback]
    Let $\begin{tikzcd}[cramped, sep=small] A \arrow[r, "f"] & C & B \arrow[l, "g"'] \end{tikzcd}$ be a cospan in $\mathbf{C}$. Its \textbf{pullback} is an object, denoted $A \times_C B$, along with morphisms $p_A:A\times_C B \rightarrow A$ and $p_B:A\times_C B \rightarrow B$ such that $f\circ p_A= g \circ p_B$ and the following universal property holds: for any object $X$ and morphisms $s: X \rightarrow A$ and $t: X \rightarrow B$ satisfying $f \circ s = g \circ t$, there is a unique morphism $!:X \rightarrow A\times_C B$ such that \eqref{diag-pullback} commutes.
    \begin{equation}\label{diag-pullback}
        \begin{tikzcd}
            X \arrow[rdd, "s"', bend right] \arrow[rrd, "t", bend left] \arrow[rd, "!", dashed] &  &  \\
                 & A\times_C B \arrow[d, "p_A"'] \arrow[r, "p_B"] \arrow[dr, phantom, "\pullback", very near start] & B \arrow[d, "g"] \\
                 & A \arrow[r, "f"'] & C
        \end{tikzcd}
    \end{equation}
    The $\pullback$ symbol is a standard convention to specify that the square is not only commutative, but also a pullback square.
\end{defn}
\begin{exmp}[\textbf{Set}]
    Let $\begin{tikzcd}[cramped, sep=small] A \arrow[r, "f"] & C & B \arrow[l, "g"'] \end{tikzcd}$ be a cospan in \textbf{Set} and suppose that its pullback is $\begin{tikzcd}[cramped, sep=small] A & A \times_C B \arrow[l, "p_A"'] \arrow[r, "p_B"]& B\end{tikzcd}$. Observe that $p_A$ and $p_B$ look like projections, and in fact, by the universality of the product $A \times B$, there is a map $h: A\times_C B \rightarrow A \times B$ such that $h(x) = (p_A(x), p_B(x))$. Consider the image of $h$, if $(a,b) \in \im(h)$, then there exists $x \in A \times_C B$ such that $p_A(x) = a$ and $p_B(x) = b$. Moreover, the commutativty of the square in \eqref{diag-pullback} implies $f(a) = g(b)$, hence \[\im(h) \subseteq \{(a,b) \in A \times B \mid f(a) = g(b)\} =: E.\]
    Now, letting $X= E$, $s = \pi_A$ and $t = \pi_B$, by definition, $f \circ s = g \circ t$ hence, there is a unique $!: E \rightarrow A\times_C B$ satsifying $p_A \circ ! = \pi_A$ and $p_B \circ ! = \pi_B$. Viewing $h$ as going in the opposite direction to $!$, it is easy to see that for any $(a,b) \in E$, \[(h\circ!)(a,b) = (p_A(!(a,b)), p_B(a,b)) = (\pi_A(a,b), \pi_B(a,b)) = (a,b),\] thus $!$ has a left inverse and is injective. Assume towards a contradiction that it is not surjective, then let $y \in A\times_C B$ not be in the image of $!$ and denote $x = !(p_A(y), p_B(y))$. Define $!'$ as acting exactly like $!$ except on $(p_A(y),p_B(y))$ where it goes to $y$ instead of $x$, this contradicts the uniqueness of $!$.

    As a particular case, if the cospan is comprised of two inclusions $A \hookrightarrow C \hookleftarrow B$, then the pullback is the intersection $A \cap B$ with $p_A$ and $p_B$ being the inclusions.
\end{exmp}
\begin{exmps}
    \textbf{In a posetal category}: the commutativy of the square in \eqref{diag-pullback} does not depend on the morphisms, thus the universal property is equivalent to the property of being a product.
\end{exmps}

\subsection{Pushout}
\begin{defn}[Span]
    A \textbf{span} in $\mathbf{C}_0$ is comprised of three objects $A,B,C$ and two morphisms $f$ and $g$ as follows.
    \begin{equation*}
        \begin{tikzcd}
            A & C \arrow[l, "f"'] \arrow[r, "g"]& B 
        \end{tikzcd}
    \end{equation*}
\end{defn}
\begin{defn}[Pushout]
    Let $\begin{tikzcd}[cramped, sep=small]A & C \arrow[l, "f"'] \arrow[r, "g"]& B \end{tikzcd}$ form a span in $\mathbf{C}$. Its \textbf{pushout} is an object, denoted $A +_C B$, along with morphisms $k_A:A \rightarrow A+_C B$ and $k_B:B \rightarrow A+_C B$ such that $k_A \circ f= k_B \circ g$ and the following universal property holds: for any object $X$ and morphisms $s: A \rightarrow X$ and $t: B \rightarrow X$ satisfying $s \circ f = t \circ g$, there is a unique morphism $!:A+_C B \rightarrow X$ such that \eqref{diag-pushout} commutes.
    \begin{equation}\label{diag-pushout}
        \begin{tikzcd}
            C \arrow[d, "f"'] \arrow[r, "g"] \arrow[dr, phantom, "\pushout", very near end] & B \arrow[d, "k_B"] \arrow[rdd, "t", bend left] &   \\
            A \arrow[r, "k_A"'] \arrow[rrd, "s"', bend right] & A+_CB \arrow[rd, "!", dashed]    &   \\
            &  & X
        \end{tikzcd}
    \end{equation}
\end{defn}
\begin{exmp}[\textbf{Set}]
    Let $\begin{tikzcd}[cramped, sep=small]A & C \arrow[l, "f"'] \arrow[r, "g"]& B \end{tikzcd}$ be a span in \textbf{Set} and suppose its pushout is $\begin{tikzcd}[cramped, sep=small] A \arrow[r, "k_A"] & A+_C B & B \arrow[l, "k_B"'] \end{tikzcd}$. Similarly to above, observe that $k_A$ and $k_B$ are like coprojections, so there is a unique map $!: A+ B \rightarrow A+_C B$ such that $!(a) = k_A(a)$ and $!(b) = k_B(b)$. Furthermore, for any $c \in C$, $!(f(c)) = !(g(c))$, thus 
    \[\exists c \in C, f(c)=a \text{ and } g(c) = b \implies !(a) = !(b).\]
    This is very similar to what happened for coequalizers and after working everything out, we obtain that $!:A+B \rightarrow A +_C B$ is the coequalizer of $\kappa_A \circ f$ and $\kappa_B \circ g$. This is a general fact that does not only apply in \textbf{Set} but in every category with binary coproducts and coequalizers.

    As a particular case, if $C = A\cap B$ and $f$ and $g$ are simply inclusions, then $A +_C B = A\cup B$ (the non-disjoint union).
\end{exmp}
\section{Generalization}
In case you have not figured out the pattern, note that products, equalizers and pullbacks are examples of limits while coproducts, coequalizers and pushouts are examples of colimits. These six examples give quite a good idea of what it is to be a limit or colimit. Roughly, all of the definitions go as follows.
\begin{itemize}
    \item Some shape is specified for a diagram $D$ (i.e.: a discrete diagram, two parallel morphisms, a span, a cospan, etc.).
    \item The limit (resp. colimit) of $D$ is an object $L$ along with morphisms in $\Hom_C(L,O)$ (resp. $\Hom_C(O,L)$) for any object $O$ in $D$ such that combining $D$ with these morphisms yields a commutative diagram.
    \item These morphisms satisfy a universal property. More specifically, for any object $L'$ with morphisms in $\Hom_C(L',O)$ (resp. $\Hom_C(O,L')$) commuting with $D$, there is a unique $!:L'\rightarrow L$ (resp. $L \rightarrow L'$) such that combining all the morphisms with $D$ yields a commutative diagram.
\end{itemize}
The first step towards a formal generalization is to formally define a diagram.
\subsection{Definitions}
\begin{defn}[Diagram]
    A \textbf{diagram} in $\mathbf{C}$ is a functor $F:\mathbf{D}\rightsquigarrow \mathbf{C}$ where $\mathbf{D}$ is usually a small or even finite category.
\end{defn}
\begin{rem}
Diagrams are usually represented by (partially) drawing the image of $F$. All the diagrams drawn up to this point define the domain of the functor implicitly. For instance, when considering a commutative square in $\mathbf{C}$, what is actually considered is the image from a functor with codomain $\mathbf{C}$ and domain the category represented in \eqref{diag-commsquare}:
    \begin{equation}\label{diag-commsquare}
        \begin{tikzcd}
            \cdot \arrow[r] \arrow[d] & \cdot \arrow[d] \\
            \cdot \arrow[r] & \cdot
        \end{tikzcd}
    \end{equation}
    It follows trivially from this definition that functors preserve commutative diagrams.
\end{rem}
Next, notice that the morphisms given for $L$ and $L'$ have the same conditions, they form a \textbf{cone or cocone}.
\begin{defn}[Cone]
    Let $F: \mathbf{D}\rightsquigarrow \mathbf{C}$ be a diagram. A \textbf{cone} from $X$ to $F$ is an object $X \in \mathbf{C}_0$, called the \textbf{tip}, along with a family of morphisms $\left\{ \psi_Y: X \rightarrow F(Y)\right\}$ indexed by objects of $Y \in \mathbf{D}_0$ such that for any morphism $f:Y \rightarrow Z$ in $\mathbf{D}_1$, $F(f) \circ \psi_Y = \psi_Z$, i.e.: diagram \eqref{diag-cone} commutes.
    \begin{equation}\label{diag-cone}
        \begin{tikzcd}
            & X \arrow[ld, "\psi_Y"'] \arrow[rd, "\psi_Z"] &\\
            F(Y)\arrow[rr, "F(f)"] & & F(Z)
        \end{tikzcd}
    \end{equation}
\end{defn}
Often, the terminology \textit{cone over} $F$ is used. Next, the fact that the morphism $!$ keeps everything commutative can be generalized.
\begin{defn}[Morphism of cones]
    Let $F:\mathbf{D}\rightsquigarrow \mathbf{C}$ be a diagram and $\{\psi_Y: A \rightarrow F(Y)\}_{Y \in \mathbf{D}_0}$ and $\{\phi_Y: B\rightarrow F(Y)\}_{Y \in \mathbf{D}_0}$ be two cones over $F$. A \textbf{morphism of cones} from $A$ to $B$ is a morphism $g:A\rightarrow B$ in $\mathbf{C}_1$ such that for any $Y\in \mathbf{D}_0$, $\phi_Y \circ g = \psi_Y$, i.e.: \eqref{diag-morphcone} commutes.
    \begin{equation}\label{diag-morphcone}
        \begin{tikzcd}
            A \arrow[rr, "g"] \arrow[rd, "\psi_Y"'] &  & B \arrow[ld, "\phi_Y"] \\
             & F(Y) & 
        \end{tikzcd}
    \end{equation}
\end{defn}
After verifying that morphisms can be composed, the last two definitions give rise to the category of cones over a diagram $F$ which we denote $\text{Cone}(F)$. Finally, the universal property can be stated in terms of cones, thus giving the general definition of a limit. Indeed, the limit of a diagram $D$ is a cone $L$ over $D$ such that for every cone $L'$ over $D$, there is a unique cone morphism $!:L'\rightarrow L$. Equivalently, $L$ is the terminal object of $\text{Cone}(F)$.
\begin{defn}[Limit]
    Let $F:\mathbf{D} \rightsquigarrow \mathbf{C}$ be a diagram, the \textbf{limit} of $F$ (or $\mathbf{D}$) denoted $\lim F$ (or $\lim \mathbf{D}$), if it exists, is the terminal object of $\text{Cone}(F)$.
\end{defn}
\begin{rem}
    Often, $\lim F$ also designates the tip of the cone as an object in $C$ rather than the cone.
\end{rem}
\begin{exmps}
    While you can play around with the three examples of limits we have already given and make them fit in this general definition, we add to this list a trivial example and a more complex one.
    \begin{enumerate}
        \item Consider an empty diagram in $\mathbf{C}$, that is, the only functor $\emptyset$ from the empty category to $\mathbf{C}$. An cone from $X$ to $\emptyset$ is just an object $X \in \mathbf{C}_0$ as there are no objects in the diagram. Consequently, a morphism in $\text{Cone}(\emptyset)$ is simply a morphism in $\mathbf{C}$, so $\text{Cone}(\emptyset)$ is the same as the original category $\mathbf{C}$ and $\lim \emptyset$ is the terminal object of $\mathbf{C}$ if it exists.
        \item Let $X = \{x_1, \dots, x_n\}$ be a set of indeterminates (also called variables) and $k$ be a field, $k[X]$ denotes the ring of polynomials over $X$. While, we will describe a nice categorical definition of $k[X]$ in Episode 6, let us assume we know what this means and we will construct $k\llbracket X\rrbracket$, the ring of formal power series over $X$, using limits.
        
        Let $I = \langle X \rangle$ be the ideal generated by $X$, the following three key property are satisfied.
        \begin{enumerate}[a)]
            \item For any $n < m \in \N$ and $p \in k[X]/I^m$, forgetting about all terms in $p$ of degree at least $n$ yields a ring homomorphism $\pi_{m,n}: k[X]/I^m \rightarrow k[X]/I^n$.
            \item For any $n \in \N$, we can do the same thing for power series to obtain a homomorphism $\pi_{\infty,n}: k\llbracket X \rrbracket \rightarrow k[X]/I^n$.
            \item Any composition of the homomorphisms above can be seen as a single homomorphism. Namely, $\forall n < m < l \in \N \cup_{\infty}$, \[\pi_{m,n} \circ \pi_{l,m} = \pi_{l,n}.\]
        \end{enumerate}
        Consider the posetal category $(\N, \geq)$, a) and c) imply that $F(n) = k[X]/I^n$ and $F(m>n) = \pi_{m,n}$ defines a functor $F: (\N, \geq) \rightarrow \textbf{Ring}$. This can be represented as diagram \eqref{diag-formalseriessystem}.
        \begin{equation}\label{diag-formalseriessystem}
            \begin{tikzcd}
                \cdots \arrow[r] & {k[X]/I^n} \arrow[r, "{\pi_{n,n-1}}"] & \cdots \arrow[r] & {k[X]/I^2} \arrow[r, "{\pi_{2,1}}"] & {k[X]/I} \arrow[r, "{\pi_{1,0}}"] & {k[X]}
            \end{tikzcd}
        \end{equation}
        Now, using b) and c), we see that $k\llbracket X \rrbracket$ along with $\{\pi_{\infty,n}\}_{n\in \N}$ is a cone over this diagram. It is in fact the terminal cone. Let $\{p_n: R \rightarrow k[X]/I^n\}$ be another cone and $!:R \rightarrow k\llbracket X \rrbracket$ a morphism of cone. By commutativity, the coefficients of $!(r)$ must agree with $p_n(r)$ on all monomials of degree at most $n$, thus,
        \[!(r) = p_0(r) + \sum_{n > 0} p_n(r) - p_{n-1}(r).\]
        This completely determines $!$, so it is unique (existence also follows from this equation).

        The construction of this diagram from quotienting different powers of the same ideal is used in different contexts, it is called the \textbf{completion} of $k[X]$ with respect to $I$. For instance, one can define the $p$-adic integers with base ring $\Z$ and the ideal generated by $p$ for any prime $p$.
    \end{enumerate}
\end{exmps}

\subsection{Codefinitions}
Put simply, a colimit in $\mathbf{C}$ is a limit in $\op{\mathbf{C}}$.  It is suggested to spend a bit of time trying to dualize all of the previous section on your own, but we have done it for completeness.  
\begin{defn}[Cocone]
    Let $F: \mathbf{D}\rightsquigarrow \mathbf{C}$ be a diagram. A \textbf{cocone} from $F$ to $X$ is an object $X \in \mathbf{C}_0$ along with a family of morphisms $\left\{ \psi_Y: F(Y) \rightarrow X \right\}$ indexed by objects of $Y \in \mathbf{D}_0$ such that for any morphism $f:Y \rightarrow Z$ in $\mathbf{D}$, $\psi_Z \circ F(f) = \psi_Y$, i.e.: \eqref{diag-cocone} commutes.
    \begin{equation}\label{diag-cocone}
        \begin{tikzcd}
            F(Y) \arrow[rd, "\psi_Y"'] \arrow[rr, "F(f)"] & & F(Z) \arrow[ld, "\psi_Z"]\\
            & X & 
        \end{tikzcd}
    \end{equation}
\end{defn}
\begin{defn}[Morphism of cocones]
    Let $F:\mathbf{D}\rightsquigarrow \mathbf{C}$ be a diagram and $\{\psi_Y: F(Y)\rightarrow A \}_{Y \in \mathbf{D}_0}$ and $\{\phi_Y: F(Y)\rightarrow B\}_{Y \in \mathbf{D}_0}$ be two cocones. A \textbf{morphism of cocones} from $A$ to $B$ is a morphism $g:A\rightarrow B$ in $\mathbf{C}$ such that for any $Y\in \mathbf{D}_0$, $g \circ \psi_Y = \phi_Y$, i.e.: \eqref{diag-morphcocone} commutes.
    \begin{equation}\label{diag-morphcocone}
        \begin{tikzcd}
            & F(Y) \arrow[ld, "\psi_Y"'] \arrow[rd, "\phi_Y"] & \\
            A \arrow[rr, "g"] &  & B  \\
        \end{tikzcd}
    \end{equation}
\end{defn}
The category of cocones from $F$, sometimes called cones under $F$, is denoted $\text{Cocone}(F)$.
\begin{defn}[Colimit]
    Let $F:\mathbf{D} \rightsquigarrow \mathbf{C}$ be a diagram, the \textbf{colimit} of $F$ (or $\mathbf{D}$) denoted $\colim F$ (or $\colim\mathbf{D}$), if it exists, is the initial object of $\text{Cocone}(F)$.
\end{defn}
\begin{exmp}
    The colimit of the empty diagram is the initial object if it exists.
\end{exmp}
\subsection{Result}
\begin{prop}[Uniqueness]
    Let $F: \mathbf{D} \rightsquigarrow \mathbf{C}$ be a diagram, the limit (resp. colimit) of $F$, if it exists, is unique up to unique isomorphism.
\end{prop}
\begin{proof}
    This follows from the uniqueness of terminal (resp. initial) objects.
\end{proof}
\begin{rem}
    The isomorphism between two limits (also colimits) is unique when viewed as a morphism of cone. There might exists an isomorphism between the tips that is not a morphism of cone. For instance, let $A$, $B$ and $C$ be finite sets. One can check that both $A \times (B \times C)$ and $(A \times B) \times C$ are products of $\{A, B, C\}$ (with the usual projection maps). Thus, there is an isomorphism between them. It is trivial to see that, for it to be a morphism of cones, it must send $(a, (b,c))$ to $((a,b), c)$, but any other bijection between them is an isomorphism in \textbf{Set}.
    
    For this reason, the limit really consists of the whole cone, and not just of the object at the tip!
\end{rem}


%SPAN: \begin{tikzcd} A & A \times_C B \arrow[l, "p_A"'] \arrow[r, "p_B"]& B\end{tikzcd}
%COSPAN: \begin{tikzcd} A \arrow[r, "f"] & C & B \arrow[l, "g"'] \end{tikzcd}

%Examples
%Product, Coproduct, Equalizer, Coequalizer, Pullback, Pushout.
%Formalization
%Diagram, Cone, Cocone, Limits and colimits. Application to previous example + p-adic numbers. Some theorems maybe.

%Next Lecture on universal properties. Generalize the limits. Comma categories, universal morphisms. Universal properties that are not limits. 

\end{document}

