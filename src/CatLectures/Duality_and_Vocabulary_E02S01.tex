\documentclass{article}

\newcommand{\bra}[1]{\left(#1\right)}
\usepackage[activate={true,nocompatibility},final,tracking=true,kerning=true,spacing=true,factor=1100,stretch=10,shrink=10]{microtype}
\microtypecontext{spacing=nonfrench}
\usepackage{tikz}
\usepackage{tikz-cd}
\usepackage{mathpazo}
\usepackage{amsmath,amsthm,amssymb}
\usepackage{subcaption}
\usepackage{enumerate}
\usetikzlibrary{shapes}
\usetikzlibrary{positioning}
% Set up the images/graphics package
\usepackage{graphicx,float}
\setkeys{Gin}{width=\linewidth,totalheight=\textheight,keepaspectratio}
\graphicspath{{.}}


% Small sections of multiple columns
\usepackage{multicol}
\usepackage[margin=1.6in]{geometry}

%--------Theorem Environments--------
%theoremstyle{plain} --- default
\newtheorem{thm}{Theorem}
\newtheorem{cor}[thm]{Corollary}
\newtheorem{prop}[thm]{Proposition}
\newtheorem{lem}[thm]{Lemma}
\newtheorem{fact}[thm]{Fact}
\newtheorem{conj}[thm]{Conjecture}
\newtheorem{quest}[thm]{Question}
\newtheorem{claim}{Claim}

\theoremstyle{definition}
\newtheorem{defn}[thm]{Definition}
\newtheorem{defns}[thm]{Definitions}
\newtheorem{con}[thm]{Construction}
\newtheorem{exmp}[thm]{Example}
\newtheorem{jk}[thm]{Joke}
\newtheorem{exmps}[thm]{Examples}
\newtheorem{notn}[thm]{Notation}
\newtheorem{notns}[thm]{Notations}
\newtheorem{addm}[thm]{Addendum}
\newtheorem{exer}[thm]{Exercise}

\theoremstyle{remark}
\newtheorem{rem}[thm]{Remark}
\newtheorem{ans}[thm]{Answer}
\newtheorem{rems}[thm]{Remarks}
\newtheorem{warn}[thm]{Warning}
\newtheorem{sch}[thm]{Scholium}

% MACROS
\newcommand{\Mod}[1]{\ (\text{mod}\ #1)}
\newcommand{\R}{\mathbb{R}}
\newcommand{\N}{\mathbb{N}}
\newcommand{\Q}{\mathbb{Q}}
\newcommand{\F}{\mathbb{F}}
\newcommand{\Z}{\mathbb{Z}}
\newcommand{\mC}{\mathcal{C}}
\newcommand{\mG}{\mathcal{G}}
\newcommand{\mP}{\mathcal{P}}
\newcommand{\one}{\mathbb{1}}
\renewcommand{\P}{\mathbb{P}}
\DeclareMathOperator{\dist}{dist}
\DeclareMathOperator{\aut}{Aut}
\DeclareMathOperator{\gal}{Gal}
\DeclareMathOperator{\orb}{Orb}
\DeclareMathOperator{\stab}{Stab}
\DeclareMathOperator{\inn}{Inn}
\DeclareMathOperator{\spn}{Span}
\DeclareMathOperator{\out}{Out}
\DeclareMathOperator{\im}{Im}
\DeclareMathOperator{\arr}{Arr}
\DeclareMathOperator{\rk}{rk}
\DeclareMathOperator{\rcf}{rcf}
\DeclareMathOperator{\tors}{Tors}
\DeclareMathOperator{\Hom}{Hom}
\DeclareMathOperator{\ann}{Ann}
\DeclareMathOperator{\syl}{Syl}
\newcommand{\norm}[1]{\left\lVert #1 \right\rVert}
\newcommand{\inp}[2]{\left\langle #1, #2 \right\rangle}
\newcommand{\id}{\text{id}}
\newcommand{\gln}{\text{GL}_n}
\newcommand{\op}[1]{#1^{\text{op}}}

\title{Lecture 2 - Duality and More Vocabulary\vspace{-10pt}}
\date{\vspace{-30pt}\today\vspace{-15pt}}  % if the \date{} command is left out, the current date will be used
\begin{document}
\maketitle
\begin{abstract} Through the exploration of duality and the presentation of more vocabulary, further familiarity with categories and functors is built.
\end{abstract}
\section{Duality}
The concept of duality is ubiquitous throughout mathematics. It can relate two perspectives of the same object as for dual vector spaces, two complementary problems such as a maximization and a minimization linear program and even two seemingly unrelated fields like topology and logic (Stone dualities). While this vague principle of duality is the foundation of many groundbreaking results, the duality in question here is categorical duality and it is a bit more precise.

Informally, there is nothing more to say than ``Take all the diagrams in a definition/theorem, reverse the arrows and reap the benefits of the dual concept/result.`` The more formal version will follow after we first exhibit the principle in action.

Recall that, intuitively, a functor is a structure preserving transformation between categories. A simple example we gave last episode was functors between posets that were order-preserving functions. However, as a consequence, one might conclude that order-reversing functions impair the structure of a poset, which feels arbitrary. The same happens between categories representing groups because anti-homomorphisms cannot arise as functors between such categories.

There are two options to remedy this discrepancy between intuition and formalism; both have duality as a guiding principle.

%TODO: reverse the order of opposite vs contravariant.
\subsection{Contravariant functors}
By modifying last lecture's definition to require that $F(f)$ goes in the opposite direction, we obtain a \textbf{contravariant} functor. Incidentally, what we defined as a functor last week is more commonly called a \textbf{covariant} functor.
\begin{defn}[Contravariant functor]
	Let $\mathbf{C}$ and $\mathbf{D}$ be categories, a \textbf{contravariant functor} $F: \mathbf{C} \rightsquigarrow \mathbf{D}$ is a pair of maps $F_0:\mathbf{C}_0 \rightarrow \mathbf{D}_0$ and $F_1:\mathbf{C}_1 \rightarrow \mathbf{D}_1$ making diagrams \eqref{diag-contrafunc1}, \eqref{diag-contrafunc2} and \eqref{diag-contrafunc3} commute (where $F_2'$ is now induced by the definition of $F_1$ with $(f,g) \mapsto (F_1(g), F_1(f))$).\\
	\begin{minipage}{0.37\textwidth}
		\begin{equation}\label{diag-contrafunc1}
		\begin{tikzcd}
		\mathbf{C}_0 \arrow[d, "F_0"'] & \mathbf{C}_1 \arrow[d, "F_1"] \arrow[l, "s"'] \arrow[r, "t"] & \mathbf{C}_0 \arrow[d, "F_0"] \\
		\mathbf{D}_0 & \mathbf{D}_1 \arrow[l, "t"] \arrow[r, "s"'] & \mathbf{D}_0
		\end{tikzcd}
		\end{equation}
	\end{minipage}	
	\begin{minipage}{0.31\textwidth}
		\begin{equation}\label{diag-contrafunc2}
		\begin{tikzcd}
		\mathbf{C}_2 \arrow[d, "\circ_{\mathbf{C}}"'] \arrow[r, "F_2'"] & \mathbf{D}_2 \arrow[d, "\circ_{\mathbf{D}}"] \\
		\mathbf{C}_1 \arrow[r, "F_1"'] & \mathbf{D}_1
		\end{tikzcd}
		\end{equation}
	\end{minipage}
	\begin{minipage}{0.31\textwidth}
		\begin{equation}\label{diag-contrafunc3}
		\begin{tikzcd}
		\mathbf{C}_0 \arrow[d, "u_{\mathbf{C}}"'] \arrow[r, "F_0"] & \mathbf{D}_0 \arrow[d, "u_{\mathbf{D}}"] \\
		\mathbf{C}_1 \arrow[r, "F_1"'] & \mathbf{D}_1
		\end{tikzcd}
		\end{equation}
	\end{minipage}\\
	In words, $F$ must satisfy the following properties.
	\begin{enumerate}[i.]
		\item For any $A, B \in \mathbf{C}_0$, if $f \in \Hom_{\mathbf{C}}(A,B)$ then $F(f) \in \Hom_{\mathbf{D}}(F(B), F(A))$.
		\item If $f,g \in \mathbf{C}_1$ are composable, then $F(f\circ g) = F(g) \circ F(f)$.
		\item If $A \in \mathbf{C}_0$, then $u_{\mathbf{D}}(F(A)) = F(u_{\mathbf{C}}(A))$.
	\end{enumerate}
\end{defn}
\begin{exmps}
	Just like their covariant counterparts, contravariant functors are quite numerous. Here are a few simple ones, we leave you to check that they satisfy the diagrams above.
	\begin{enumerate}
		\item It is easy to verify that contravariant functors $F: (X, \leq) \rightsquigarrow (Y, \subseteq)$ correspond to \textbf{order-reversing} functions between the posets $X$ and $Y$ while contravariant functors $F: \mathbf{B}G \rightsquigarrow \mathbf{B}H$ correspond to \textbf{anti-homomorphisms} between the groups $G$ and $H$.
		\item The contravariant powerset functor $\widehat{\mP}: \textbf{Set} \rightsquigarrow \textbf{Set}$ sends a set $X$ to its power set $\mP(X)$ and a function $f: X\rightarrow Y$ to the pre-image map $\widehat{\mP}(f):\mP(Y)\rightarrow \mP(X)$, the latter sends a subset $S\subseteq Y$ to \[f^{-1}(S) = \{x \in X \mid f(x) \in S\} \subseteq X.\]
	\end{enumerate}
\end{exmps}
Next, there is a couple of functors that are key to understand the philosophy put forward by category theory (we will talk more about it when covering the Yoneda lemma in Episode 6).
\begin{exmp}[Hom functors]
	Let $\mathbf{C}$ be a locally-small category and $A \in \mathbf{C}_0$ one of its object. We define the covariant and contravariant $\Hom$ functors from $\mathbf{C}$ to $\textbf{Set}$.
	\begin{enumerate}
		\item The functor $\Hom_{\mathbf{C}}(A,-): \mathbf{C} \rightsquigarrow \textbf{Set}$ sends an object $B\in \mathbf{C}_0$ to the Hom-set $\Hom_{\mathbf{C}}(A,B)$ and a morphism $f:B\rightarrow B'$ to the function $$\Hom_{\mathbf{C}}(A,f): \Hom_{\mathbf{C}}(A,B) \rightarrow \Hom_{\mathbf{C}}(A,B') = g \mapsto f\circ g.$$
		This function is called \textbf{post-composition by $f$} and is denoted $f \circ (-)$ or $f_{*}$. Let us show $\Hom_{\mathbf{C}}(A, -)$ is a covariant functor.
		\begin{enumerate}[i.]
			\item For any $f \in \mathbf{C}_1$, it is clear from the definitions that \[\Hom_{\mathbf{C}}(A,s(f)) = s(\Hom_{\mathbf{C}}(A,f)) \text{ and } \Hom_{\mathbf{C}}(A,t(f)) = t(\Hom_{\mathbf{C}}(A,f)).\]
			\item For any $(f_1,f_2) \in \mathbf{C}_2$, we claim that \[\Hom_{\mathbf{C}}(A,f_1\circ f_2) = \Hom_{\mathbf{C}}(A,f_1)\circ \Hom_{\mathbf{C}}(A,f_2).\] In the L.H.S., an element $g \in \Hom_{\mathbf{C}}(A,s(f_1\circ f_2))$ is mapped to $(f_1 \circ f_2) \circ g$ and in the R.H.S., an element $g \in \Hom_{\mathbf{C}}(A,s(f_2)$ is mapped to $f_1\circ (f_2 \circ g)$. Since $s(f_1 \circ f_2) = s(f_2)$ and composition is associative, we conclude that the two maps are the same.
			\item For any $B \in \mathbf{C}_0$, the post-composition by $u_{\mathbf{C}}(B)$ is defined to be the identity, hence the third diagram also commutes.
		\end{enumerate}
		\item The functor $\Hom_{\mathbf{C}}(-,A): \mathbf{C} \rightsquigarrow \textbf{Set}$ sends an object $B\in \mathbf{C}_0$ to the Hom-set $\Hom_{\mathbf{C}}(B,A)$ and a morphism $f:B\rightarrow B'$ to the function $$\Hom_{\mathbf{C}}(f,A): \Hom_{\mathbf{C}}(B',A) \rightarrow \Hom_{\mathbf{C}}(B,A) = g \mapsto g\circ f.$$
		This function is called \textbf{pre-composition by $f$} and is denoted $(-) \circ f$ or $f^*$. Let us show $\Hom_{\mathbf{C}}(-,A)$ is a contravariant functor.
		\begin{enumerate}[i.]
			\item For any $f \in \mathbf{C}_1$, it is clear from the definitions that \[\Hom_{\mathbf{C}}(s(f),A) = t(\Hom_{\mathbf{C}}(f,A))\text{ and } \Hom_{\mathbf{C}}(t(f),A) = s(\Hom_{\mathbf{C}}(f,A)).\]
			\item For any $(f_1,f_2) \in \mathbf{C}_2$, we claim that \[\Hom_{\mathbf{C}}(f_1\circ f_2,A) = \Hom_{\mathbf{C}}(f_2,A)\circ \Hom_{\mathbf{C}}(f_1,A).\] In the L.H.S., an element $g \in \Hom_{\mathbf{C}}(t(f_1\circ f_2),A)$ is mapped to $g\circ (f_1 \circ f_2)$ and in the R.H.S., an element $g \in \Hom_{\mathbf{C}}(t(f_1),A)$ is mapped to $(g\circ f_1) \circ f_2$. Since $t(f_1 \circ f_2) = t(f_1)$ and composition is associative, we conclude that the two maps are the same.
			\item For any $B \in \mathbf{C}_0$, pre-composition by $u_{\mathbf{C}}(B)$ is defined to be the identity, hence the third diagram also commutes.
		\end{enumerate}
	\end{enumerate}
\end{exmp}
\subsection{Opposite Category}
Another way to deal with order-reversing maps $(X, \leq) \rightarrow (Y, \subseteq)$ is to consider the reverse order on $X$ and a covariant functor $(X, \geq) \rightsquigarrow (Y, \subseteq)$. This also works for anti-homomorphims by constructing the opposite group $\op{G}$ in which the operation is reversed, namely $g\op{\cdot} h = hg$. The opposite category is a generalization of these constructions.

\begin{defn}[Opposite category]
	Let $\mathbf{C}$ be a category, we denote the \textbf{opposite category} with $\op{\mathbf{C}}$ and define it by 
	\[ \op{\mathbf{C}}_0 = \mathbf{C}_0,\ \op{\mathbf{C}}_1 = \mathbf{C}_1,\ \op{s} = t,\ \op{t} = s,\ u_{\op{\mathbf{C}}} = u_{\mathbf{C}}\]
	with the composition defined by $\op{f}\op{\circ}\op{g} = \op{(g\circ f)}$.\footnote{Note that the $\op{}$ notation here is just used to distinguish elements in $\mathbf{C}$ and $\op{\mathbf{C}}$ but the class of objects and morphisms are the same.} This canonically leads to the following contravariant functor $\op{(-)}_{\mathbf{C}}: \mathbf{C} \rightsquigarrow \op{\mathbf{C}}$ which sends an object $A$ to $\op{A}$ and a morphism $f$ to $\op{f}$. It is called the \textbf{opposite functor}.
\end{defn}
\begin{rem}
	With this definition, one can see contravariant functors as covariant functors. Formally, let $F:\mathbf{C}\rightsquigarrow \mathbf{D}$ be a contravariant functor, we can view $F$ as covariant functor from $\op{\mathbf{C}}$ to $\mathbf{D}$ or from $\mathbf{C}$ to $\op{\mathbf{D}}$ via the compositions $F\circ \op{(-)}_{\op{\mathbf{C}}}$ and $\op{(-)}_{D}\circ F$ respectively. It also follows that if $F$ and $G$ are composable functors, then $F\circ G$ is contravariant whenever exactly one of them is contravariant.
\end{rem}

\begin{exmps}
	As hinted at before, the category corresponding to $(X, \geq)$ is the opposite category of $(X, \leq)$ and $\op{(\mathbf{B}G)}$ is the category corresponding to the opposite group of $G$. While there are other interesting examples, the opposite construction is usually used implicitly to avoid dealing with contravariant functors or to avoid proving the dual of an already proven result.
	
	%TODO: example of dual vector space
\end{exmps}


Let us continue to illustrate how duality can be useful with some simple definitions and results.
\begin{defn}[Monomorphism]
	Let $\mathbf{C}$ be a category, a morphism $f \in \mathbf{C}_1$ is said to be \textbf{monic} (or a \textbf{monomorphism}) if for any $(f,g), (f,h) \in \mathbf{C}_2$ where $g$ and $h$ have the same source, $f\circ g = f\circ h$ implies $g = h$. Equivalently, $f$ is monic if $g = h$ whenever the following diagram commutes.
	\begin{equation}
		\begin{tikzcd}
		\bullet \arrow[r, "h"', bend right] \arrow[r, "g", bend left] & \bullet \arrow[r, "f"] & \bullet
		\end{tikzcd}
	\end{equation}
	
	Standard notation for a monomorphism is $ \bullet \hookrightarrow \bullet $.
\end{defn}
\begin{prop}\label{propmon1}
	Let $\mathbf{C}$ be a category and $f:A\rightarrow B$ a morphism, if there exists $f': B\rightarrow A$ such that $f'\circ f = \id_A$, then $f$ is a monomorphism.
\end{prop}
\begin{proof}
	If $f\circ g = f\circ h$, then $f'\circ f \circ g = f'\circ f \circ h$ implying $g = h$.
\end{proof}
A monomorphism with a left inverse is called a \textbf{split monomorphism}.
\begin{prop}\label{propmon2}
	Let $\mathbf{C}$ be a category and $(f_1, f_2) \in \mathbf{C}_2$, if $f_1 \circ f_2$ is a monomorphism, then $f_2$ is a monomorphism.
\end{prop}
\begin{proof}
	Let $g,h \in \mathbf{C}_1$ be such that $f_2\circ g = f_2\circ h$, we immediately get that $(f_1\circ f_2)\circ g = (f_1 \circ f_2) \circ h$. Since $f_1\circ f_2$ is a monomorphism, this implies $g = h$.
\end{proof}
The last two results make it obvious that monomorphisms are analogous to injective functions and we will see that they are exactly the same in the category \textbf{Set}, but first let us introduce the \textbf{dual concept}. Given a definition or statement in an arbitrary category $\mathbf{C}$, one could view this concept inside the category $\op{\mathbf{C}}$ and obtain a similar definition or statement where all arrows are reversed, this is called the dual concept. Dualizing the definition of a monomorphism yields an epimorphism.
\begin{defn}[Epimorphism]
	Let $\mathbf{C}$ be a category, a morphism $f \in \mathbf{C}_1$ is said to be \textbf{epic} (or an \textbf{epimorphism}) if for any two morphisms $(g,f), (h,f) \in \mathbf{C}_2$ where $g$ and $h$ have the same target, $g\circ f = h\circ f$ implies $g = h$. Equivalently, $f$ is epic if $g = h$ whenever the following diagram commutes.
	\begin{equation}
	\begin{tikzcd}
	\bullet \arrow[r, "f"] & \bullet \arrow[r, "g", bend left] \arrow[r, "h"', bend right] & \bullet
	\end{tikzcd}
	\end{equation}
	Standard notation for an epimorphism is $ \bullet \twoheadrightarrow \bullet$.
\end{defn}
The dual versions of Propositions \ref{propmon1} and \ref{propmon2} also hold. Although doing the straightforward proofs is very easy, the two next proofs rely on duality and convey the general sketch that works anytime a dual result needs to be proven.
\begin{prop}\label{propep1}
	Let $\mathbf{C}$ be a category and $f:A\rightarrow B$ a morphism, if there exists $f': B\rightarrow A$ such that $f\circ f' = \id_B$, then $f$ is epic.
\end{prop}
\begin{proof}
	Observe that $f$ is epic in $\mathbf{C}$ if and only if $\op{f}$ is monic in $\op{\mathbf{C}}$ (reverse the arrows in the definition).\footnote{This is one other way to see that two concepts are dual.} Moreover, by definition, \[\op{f'} \circ \op{f} = \op{(f \circ f')} = \op{\id_B} = \id_{\op{B}},\] so by the result for monomorphisms, $\op{f}$ is monic and hence $f$ is epic. 
\end{proof}
An epimorphism with a right inverse is called a \textbf{split epimorphism}.
\begin{prop}
	Let $\mathbf{C}$ be a category and $(f_1, f_2) \in \mathbf{C}_2$, if $f_1 \circ f_2$ is epic, then $f_2$ is epic.
\end{prop}
\begin{proof}
	Since $\op{f_2} \circ \op{f_1} = \op{(f_1 \circ f_2)}$ is monic, the result for monomorphisms implies $\op{f_2}$ is monic and hence $f_2$ is epic.
\end{proof}
\begin{exmp}[\textbf{Set}]
	\begin{itemize}
		\item[]
		\item A function $f:A\rightarrow B$ is a monomorphism if and only if it is injective:
		
		($\Leftarrow$) Since $f$ is injective, it has a left inverse, so it is monic by Proposition \ref{propmon1}.
		
		($\Rightarrow$) Given $a \in A$, let $g_a: \mathbf{1}:=\{\ast\} \rightarrow A$ be the function sending $\ast$ to $a$. For any $a_1 \neq a_2 \in A$, the functions $g_{a_1}$ and $g_{a_2}$ are different, hence $f \circ g_{a_1} \neq f \circ g_{a_2}$. Therefore, $f(a_1) \neq f(a_2)$ and since $a_1$ and $a_2$ were arbitrary, $f$ is injective.
		
		\item A function $f:A\rightarrow B$ is an epimorphism if and only if it is surjective:
		
		($\Leftarrow$) Since $f$ is surjective, it has a right inverse, so it is epic by Proposition \ref{propep1}.
		
		($\Rightarrow$) Let $h: B \rightarrow \{0,1\}=:\mathbf{2}$ be the constant function at $1$ and $g:B \rightarrow \mathbf{2}$ be the indicator function of $\im(f) \subseteq B$, namely, \[g(x) = \begin{cases}1&\exists a \in A, x = f(a)\\0&\text{o/w}\end{cases}.\]
		It is clear that $g \circ f = h\circ f \equiv 1$ and since $f$ is epic, it implies $g = h$. Thus, any element of $B$ is in the image of $f$, that is $f$ is surjective.
	\end{itemize}
\end{exmp}

\begin{exmp}[\textbf{Mon}]
Inside the category \textbf{Mon} where objects are monoids and morphims are monoid homomorphisms, the monomorphisms correspond exactly to injective homomorphims.

($\Rightarrow$) Let $f:M\rightarrow M'$ be an injective homomorphims and $g_1,g_2:N\rightarrow M$ be two parallel homomorphisms. Suppose that $f\circ g_1 = f\circ g_2$, then for all $x \in N$, $f(g_1(x)) = f(g_2(x))$, so by injectivity of $f$, $g_1(x) = g_2(x)$. Therefore $g_1 = g_2$ and since $g_1$ and $g_2$ were arbitrary, $f$ is a monomorphism.

($\Leftarrow$) Let $f:M\rightarrow M'$ be a monomorphism. Let $x,y \in M$ and define $p_x :\N \rightarrow M$ by $k\mapsto x^k$ and similarly for $p_y$. It is trivial to show that $p_x$ and $p_y$ are homomorphism. If $f(x) = f(y)$, then, by the homomorphism property, for all $k \in \N$
\[f(p_x(k))= f(x^k) = f(x)^k  = f(y)^k = f(y^k) = f(p_y(k)).\]
In other words, we get $f\circ p_x = f \circ p_y$, so $p_x = p_y$ and $x = y$. This direction follows.

Conversely, an epimorphism is not necessarily surjective. For example, the inclusion homomorphism $i:\N \rightarrow \Z$ is clearly not surjective but it is an epimorphism. Indeed, let $g,h: \Z\rightarrow M$ be two monoid homomorphisms satisfying $g \circ i = h\circ i$. In particular, $g(n) = h(n)$ for any $n \in \N\subset \Z$. It remains to show that also $g(-n) = h(-n)$: we have $ g(0) = g(n-n) = g(n) + g(-n) = 0 = h(n) + h(-n) $, but then $ h(-n) = g(-n) + g(n) + h(-n) = g(-n) $.
\end{exmp}
%TODO: example in cat: monic -> embedding. epic -> ?
\begin{defn}[Isomorphism]
	Let $\mathbf{C}$ be a category, a morphism $f:A\rightarrow B$ is said to be an \textbf{isomorphism} if there exists a morphism $f^{-1}: B\rightarrow A$ such that $f\circ f^{-1} = \id_B$ and $f^{-1}\circ f = \id_A$.
\end{defn}
\begin{rem}
	The results shown about monic and epic morphisms imply that any isomorphism is monic and epic. However, the converse is not true as witnessed by the inclusion morphism $i$ described in the example above. If there exists an isomorphism between two objects $A$ and $B$, then they are \textbf{isomorphic}, denoted $A \cong B$. Isomorphic objects are also isomorphic in the opposite category as is easy to verify, that is, the concept of isomorphism is \textit{self-dual}.
\end{rem}
\begin{defn}[Initial object]
	Let $\mathbf{C}$ be a category, an object $A \in \mathbf{C}_0$ is said to be \textbf{initial} if for any $B \in \mathbf{C}_0$, $|\Hom_{\mathbf{C}}(A,B)| = 1$, namely there are no two parallel morphisms with source $A$ and every object has a morphism coming from $A$.
\end{defn}
\begin{defn}[Terminal object]
	Let $\mathbf{C}$ be a category, an object $A \in \mathbf{C}_0$ is said to be \textbf{terminal} (or \textbf{final}) if for any $B \in \mathbf{C}_0$, $|\Hom_{\mathbf{C}}(B,A)| = 1$, namely there are no two parallel morphisms with target $A$ and every object has a morphism going to $A$.
\end{defn}
\begin{rem}[Notation]
	The terminal object of a category is often denoted $\mathbf{1}$, for any object $X$, the \textit{unique} arrow into $\mathbf{1}$ is denoted $[]:X \rightarrow \mathbf{1}$ and the unique arrow from an initial object $I$ is denoted $(): I \rightarrow X$. The motivation behind these notations is given in the next episode.
\end{rem}
It is clear that an object is initial in a category $\mathbf{C}$ if and only if it is terminal in $\op{\mathbf{C}}$. Also, if an object is initial and terminal, we say it is a \textbf{zero} object and usually denote it $\mathbf{0}$.
\begin{exmps}
	Here are examples of categories where initial and terminal objects may or may not exist.
	\begin{enumerate}
		\item $\exists$ terminal, $\nexists$ initial: Consider the poset $(\N, \geq)$ represented by the diagram below. It is clear that $0$ is terminal and no element can be initial because $0 \geq x$ implies $x = 0$.
		\begin{equation}
			\begin{tikzcd}
			\stackrel{0}{\bullet}  & \arrow[l] \stackrel{1}{\bullet}  & \arrow[l] \stackrel{2}{\bullet}  & \arrow[l] \cdots
			\end{tikzcd}
		\end{equation}
		\item %TODO: change this for the field with characteristic p or 0.
		$\nexists$ terminal, $\exists$ initial: The category \textbf{FinGrpInj} where the objects are finite groups and the morphisms are injective homomorphisms only contains an initial object $\{1\}$. Indeed, an injective homomorphism $G \rightarrow H$ can be seen as subgroup of $H$ isomorphic to $G$. The identity group $\{1\}$ can only be isomorphic to the the identity subgroup as any other element has degree more than 1, so $\{1\}$ is initial. Moreover, a group $G$ cannot be terminal as $G \times (\Z/2\Z)$ cannot be isomorphic to any subgroup of $G$.
		\item $\nexists$ terminal, $\nexists$ initial: Let $G$ be a non trivial group. The category $\mathbf{B}G$ has a single object $*$ with $\Hom_{\mathbf{B}G}(*, *) = G$ and the composition rule being the multiplication in $G$. The only object $*$ cannot be initial nor terminal as $|\Hom_{\mathbf{B}G}(*,*)| > 1$.
		
		The category whose objects are fields and morphisms are field homomorphisms also has no initial nor terminal objects because there are no field homomorphisms between fields of different characteristics.
		\item $\exists$ terminal, $\exists$ initial: Let $X$ be a non-empty topological space where $\tau$ is the collection of open sets (recall that it must contain $\emptyset$ and $X$). Let $T_X$ be the category representing the poset of open sets with inclusion as the relation. Namely, the objects are the open sets and for any two open sets $U, V \in \tau$, 
		\[\Hom_{T_X}(U,V) = \begin{cases}i_{U,V} & U \subseteq V\\ \emptyset & U \not\subseteq V\end{cases}\]
		Since the empty set is contained in every open set, it is an initial object. Since the full set $X$ contains every open set, it is a terminal object. No other set can be initial as it cannot be contained in $\emptyset$ nor be terminal as it cannot contain $X$. Moreover, note that the two objects are  not isomorphic as $\Hom_{T_X}(X, \emptyset) = \emptyset$.
	\end{enumerate}
\end{exmps}
\begin{prop}
	Let $\mathbf{C}$ be a category and $A,B\in \mathbf{C}_0$ be initial, then $A \cong B$.
\end{prop}
\begin{proof}
	Let $f$ be the single element in $\Hom_{\mathbf{C}}(A,B)$ and $f'$ be the single element in $\Hom_{\mathbf{C}}(B,A)$. Since the identity morphisms are the only elements of $\Hom_{\mathbf{C}}(A,A)$ and $\Hom_{\mathbf{C}}(B,B)$, $f' \circ f$ and $f\circ f'$, belonging to these sets, must be the identities. In other words $f$ and $f'$ are inverses, thus $A \cong B$. 
\end{proof}
The dual result follows.
\begin{prop}
	Let $\mathbf{C}$ be a category and $A,B \in \mathbf{C}_0$ be terminal, then $A \cong B$.
\end{prop}
Moreover, initial (resp. terminal) objects are unique up to \textit{unique} isomorphisms.%TODO: Note that the uniqueness of isomorphism is stronger than a simple isomorphism (try to give more intuition about this with an example)

\begin{exmp}
For our last example of duality in this lecture, let $X$ be a set and consider the posetal category $(\mP(X), \subseteq)$. We would like to define the union of two subsets of $X$ in this category. The usual definition $A \cup B = \{x \in X \mid x \in A \text{ or } x \in B\}$ is not suitable, because the data in the posetal category $\mP(X)$ never refers to elements of $X$. In particular, the subsets $A,B \subseteq X$ are simply objects in the category and it is not clear to us how we can determine what elements are in $A$ and $B$ with our categorical tools (objects and morphisms).

We propose another characterisation of the union of $A$ and $B$. First, what is obvious, $A \cup B$ contains $A$ and it contains $B$. Second, $A \cup B$ is the smallest subset of $X$ containing $A$ and $B$. Indeed, if $Y \subseteq X$ contains all element in $A$ and $B$, then it also contains $A \cup B$. Using the order $\subseteq$ (or equivalently, the morphisms in the category $\mP(X)$), we have $A, B \subseteq A\cup B$ and $\forall Y \text{ s.t. } A, B \subseteq Y \text{ then } A\cup B \subseteq Y$. We leave it as an exercise to show that $A \cup B$ is the only subset of $X$ satisfying this property.

The dual of this property (reversing all inclusions) is as follows ($\square$ is a placeholder for the operation which we will find to be dual to union).
\[ A \square B \subseteq A, B \text{ and } \forall Y \text{ s.t. } Y \subseteq A,B \text{ then } Y \subseteq A \square B\]
Putting this in words, $A \square B$ is the largest subset of $X$ which is contained in $A$ and $B$. That is, of course, the intersection $A \cap B$. In this way, union and intersection are dual operations. If you search your memory for properties about union and intersection that you proved when you first learned about sets, you will find that they usually come in pairs; the first property being the dual of the second.
\end{exmp}
\end{document}

