\documentclass{article}

\newcommand{\bra}[1]{\left(#1\right)}
\usepackage[activate={true,nocompatibility},final,tracking=true,kerning=true,spacing=true,factor=1100,stretch=10,shrink=10]{microtype}
\microtypecontext{spacing=nonfrench}
\usepackage{tikz}
\usepackage{tikz-cd}
\usepackage{mathpazo}
\usepackage{amsmath,amsthm,amssymb}
\usepackage{subcaption}
\usepackage{enumerate}
\usetikzlibrary{shapes}
\usetikzlibrary{positioning}
\usetikzlibrary{decorations.pathmorphing}
% Set up the images/graphics package
\usepackage{graphicx,float}
\setkeys{Gin}{width=\linewidth,totalheight=\textheight,keepaspectratio}
\graphicspath{{.}}


% Small sections of multiple columns
\usepackage{multicol}
\usepackage[margin=1.6in]{geometry}

%--------Theorem Environments--------
%theoremstyle{plain} --- default
\newtheorem{thm}{Theorem}
\newtheorem{cor}[thm]{Corollary}
\newtheorem{prop}[thm]{Proposition}
\newtheorem{lem}[thm]{Lemma}
\newtheorem{fact}[thm]{Fact}
\newtheorem{conj}[thm]{Conjecture}
\newtheorem{quest}[thm]{Question}
\newtheorem{claim}{Claim}

\theoremstyle{definition}
\newtheorem{defn}[thm]{Definition}
\newtheorem{defns}[thm]{Definitions}
\newtheorem{con}[thm]{Construction}
\newtheorem{exmp}[thm]{Example}
\newtheorem{jk}[thm]{Joke}
\newtheorem{exmps}[thm]{Examples}
\newtheorem{notn}[thm]{Notation}
\newtheorem{notns}[thm]{Notations}
\newtheorem{addm}[thm]{Addendum}
\newtheorem{exer}[thm]{Exercise}

\theoremstyle{remark}
\newtheorem{rem}[thm]{Remark}
\newtheorem{ans}[thm]{Answer}
\newtheorem{rems}[thm]{Remarks}
\newtheorem{warn}[thm]{Warning}
\newtheorem{sch}[thm]{Scholium}

% MACROS
\newcommand{\Mod}[1]{\ (\text{mod}\ #1)}
\newcommand{\R}{\mathbb{R}}
\newcommand{\N}{\mathbb{N}}
\newcommand{\Q}{\mathbb{Q}}
\newcommand{\F}{\mathbb{F}}
\newcommand{\Z}{\mathbb{Z}}
\newcommand{\mC}{\mathcal{C}}
\newcommand{\mG}{\mathcal{G}}
\newcommand{\mP}{\mathcal{P}}
\newcommand{\one}{\mathbb{1}}
\renewcommand{\P}{\mathbb{P}}
\DeclareMathOperator{\dist}{dist}
\DeclareMathOperator{\aut}{Aut}
\DeclareMathOperator{\gal}{Gal}
\DeclareMathOperator{\orb}{Orb}
\DeclareMathOperator{\stab}{Stab}
\DeclareMathOperator{\inn}{Inn}
\DeclareMathOperator{\spn}{Span}
\DeclareMathOperator{\out}{Out}
\DeclareMathOperator{\im}{Im}
\DeclareMathOperator{\arr}{Arr}
\DeclareMathOperator{\rk}{rk}
\DeclareMathOperator{\rcf}{rcf}
\DeclareMathOperator{\tors}{Tors}
\DeclareMathOperator{\Hom}{Hom}
\DeclareMathOperator{\ann}{Ann}
\DeclareMathOperator{\syl}{Syl}
\newcommand{\norm}[1]{\left\lVert #1 \right\rVert}
\newcommand{\inp}[2]{\left\langle #1, #2 \right\rangle}
\newcommand{\id}{\text{id}}
\newcommand{\gln}{\text{GL}_n}
\newcommand{\op}[1]{#1^{\text{op}}}

\title{Episode 5 - Natural Transformations\vspace{-10pt}}
\author{Ralph Sarkis}
\date{\vspace{-10pt}\today\vspace{-15pt}}  % if the \date{} command is left out, the current date will be used
\begin{document}
\maketitle
\begin{abstract} In this episode, we climb one more level of the tower of abstraction mentioned in the last paragraph of the first episode. After introducing natural transformations, we discuss 2-categories and equivalences.
\end{abstract}
\section{Natural Transformations}
Natural transformations are admittedly what made mathematicians want to study category theory in the first place. In short, they are morphisms between functors, i.e.: transformations that preserve the structure of functors.

The abstract structure of a category is very familiar because it resembles what is found in algebraic structures such as groups, rings or vectors spaces. That is to say, it consists of the data of one or more sets with one or more operations satisfying one or more properties. In contrast, the definition of a functor is more opaque and by itself, the structure of a functor is not obvious. A functor is effectively a morphism between categories, hence a natural transformation will be a \textit{morphism between morphisms}. Before moving on, one might find it enlightening to look for a satisfying definition of morphism between two group homomorphisms $f,g: G \rightarrow H$ and then observe its meaning when $f$ and $g$ are seen as functors $\mathbf{B}G \rightsquigarrow \mathbf{B}H$.

For the general case, let $F,G: \mathbf{C}\rightsquigarrow \mathbf{D}$ be functors. Morally, the structure of $F$ and $G$ is encapsulated in the following diagrams for every arrow, $f \in \Hom_{\mathbf{C}}(A,B)$.
\begin{center}
	
	\begin{minipage}{0.38\textwidth}
		\begin{equation}\label{diag-funcF}
		\begin{tikzcd}
		A \arrow[d, "f"'] \arrow[r, "F_0"] & F(A) \arrow[d, "F_1(f)"] \\
		B \arrow[r, "F_0"']                & F(B)                 
		\end{tikzcd}
		\end{equation}
	\end{minipage}
	\begin{minipage}{0.38\textwidth}
		\begin{equation}\label{diag-funcG}
		\begin{tikzcd}
		A \arrow[d, "f"'] \arrow[r, "G_0"] & G(A) \arrow[d, "G_1(f)"] \\
		B \arrow[r, "G_0"']                & G(B)                 
		\end{tikzcd}
		\end{equation}
	\end{minipage}
\end{center}
Thus, a morphism between $F$ and $G$ should fit in this picture by sending diagram \eqref{diag-funcF} to diagram \eqref{diag-funcG} in a commutative way.
\begin{defn}[Natural transformation]\label{defnattran}
	Let $F,G : \mathbf{C} \rightsquigarrow \mathbf{D}$ be two (covariant) functors, a \textbf{natural transformation} $\phi: F \Rightarrow G$ is a map $\phi: \mathbf{C}_0 \rightarrow \mathbf{D}_1$ that satisfies $\phi(A) \in \Hom_{\mathbf{D}}(F(A), G(A))$ for all $A \in \mathbf{C}_0$ and makes diagram \eqref{diag-nattrans} commute for any $f \in \Hom_{\mathbf{C}}(A,B)$:
	\begin{equation}\label{diag-nattrans}
	\begin{tikzcd}
	F(A) \arrow[d, "F(f)"'] \arrow[r, "\phi(A)"] & G(A) \arrow[d, "G(f)"] \\
	F(B) \arrow[r, "\phi(B)"'] & G(B)
	\end{tikzcd}
	\end{equation}
\end{defn}
As usual, there are trivial examples of natural transformation such has the identity transformation $\one_F:F \Rightarrow F$ that sends every object $A$ to the identity map $\id_{F(A)}$, but let us go back to the group case. Although very specific to single object categories, it is simple enough to quickly digest.
\begin{exmp}\label{exmp-grouphom}
	Let $f,g: \mathbf{B}G \rightsquigarrow \mathbf{B}H$ be functors (i.e.: group homomorphisms), both send the unique object $*$ in $\mathbf{B}G$ to $*$ in $\mathbf{B}H$. Thus, a natural transformation $\phi : f\Rightarrow g$ assigns to the object $*$ a morphism $* \rightarrow *$ in $H$, which is simply an element $\phi \in H$. The commutativity condition is then exhibited by diagram \eqref{diag-homnattrans} (which lives in $\mathbf{B}H$) for any $x \in G$ .
	\begin{equation}\label{diag-homnattrans}
	\begin{tikzcd}
	* \arrow[d, "f(x)"'] \arrow[r, "\phi"] & * \arrow[d, "g(x)"] \\
	* \arrow[r, "\phi"'] & *
	\end{tikzcd}
	\end{equation}
	Recall that composition in $\mathbf{B}H$ is just multiplication in $H$, so naturality of $\phi$ says that for any $x \in G$, $\phi \cdot f(x) = g(x) \cdot \phi$. Equivalently, $\phi f(x) \phi^{-1} = g(x)$. Therefore, $g = c_{\phi} \circ f$ where $c_{\phi}$ denotes conjugation by $\phi$. In conclusion, natural transformations between group homomorphisms correspond to factorizations through conjugations.
\end{exmp}
Next, an example closer to the general idea of a natural transformation.
\begin{exmp}
	Here, \textbf{CRing} will denote the category of commutative rings and \textbf{Grp} the category of groups. Fix some $n \in \N$ and define the functor $\gln:\textbf{CRing} \rightsquigarrow \textbf{Grp}$ by 
	\begin{align*}
	R &\mapsto \gln(R) \mbox{ for any commutative ring $R$ and} \\
	f &\mapsto \gln(f) \mbox{ for any ring homomorphism $f$.}
	\end{align*}
	The map $\gln(f)$ is just the extension of $f$ on $\gln(R)$ by applying $f$ to every element of the matrices. The second functor is $(-)^{\times}:\textbf{CRing} \rightsquigarrow \textbf{Grp}$ which sends a commutative ring $R$ to its group of units $R^{\times}$ under multiplication and a ring homomorphism $f$ to $f^{\times}$, its restriction on $R^{\times}$. Checking these mappings define two covariant functors is left as an (simple) exercise, but one might expect these to be functors as they play nicely with the structure of the objects involved.
	
	A natural transformation between these two functors is $\det:\gln \Rightarrow (-)^{\times}$ which maps a commutative ring $R$ to $\det_R$, the function calculating the determinant of a matrix in $\gln(R)$. The first thing to check is that $\det_R \in \Hom_{\textbf{Grp}}(\gln(R), R^{\times})$ which is clear because the determinant of an invertible matrix is always a unit, $\det(I_n) = 1$ and $\det$ is a multiplicative map. The second thing is to verify that diagram \eqref{diag-detnat} commutes for any $f\in \Hom_{\textbf{CRing}}(R,S)$:
	\begin{equation}\label{diag-detnat}
	\begin{tikzcd}
	\text{GL}_n(R) \arrow[r, "\det_R"] \arrow[d, "\text{GL}_n(f)"'] & R^{\times} \arrow[d, "f^{\times} = f\mid_{R^{\times}}"] \\
	\text{GL}_n(S) \arrow[r, "\det_S"'] & S^{\times}
	\end{tikzcd}
	\end{equation}
	We will check the claim for $n=2$, but the general proof should only involve more notation to write the bigger expressions, no novel idea. Let $a,b,c,d \in R$, we have 
	\begin{align*}
	(\det{}_S \circ \text{GL}_2(f))\left( \begin{bmatrix}a&b\\c&d\end{bmatrix} \right)&= 
	\det{}_S\left(\begin{bmatrix}f(a)&f(b)\\f(c)&f(d)\end{bmatrix}\right)\\
	&= f(a)f(d)-f(b)f(c)\\
	&= f(ad-bc)\\
	&= f^{\times}(ad-bc)\\
	&= (f^{\times}\circ \det{}_R)\left( \begin{bmatrix}a&b\\c&d\end{bmatrix}\right).
	\end{align*}
	We conclude that the diagram commutes and that $\det$ is indeed a natural transformation.
\end{exmp}
Now, in order to talk about a category of functors, it remains to describe the composition of natural transformations.
\begin{defn}[Vertical composition]
	Let $F,G,H: \mathbf{C}\rightsquigarrow \mathbf{D}$ be parallel functors and $\phi:F\Rightarrow G$ and $\eta:G\Rightarrow H$ be two natural transformations. Then, the \textbf{vertical composition} of $\phi$ and $\eta$, denoted $\eta\cdot \phi:F\Rightarrow H$ is defined by $(\eta \cdot \phi)(A) = \eta(A) \circ \phi(A)$ for all $A \in \mathbf{C}_0$. If $f: A\rightarrow B$ is a morphism in $\mathbf{C}$, then diagram \eqref{diag-vertcomp} commutes by naturality of $\phi$ and $\eta$, showing that $\eta \cdot \phi$ is a natural transformation from $F$ to $H$.
	\begin{equation}\label{diag-vertcomp}
	\begin{tikzcd}
	F(A) \arrow[r, "\phi(A)"] \arrow[d, "F(f)"'] & G(A) \arrow[r, "\eta(A)"] \arrow[d, "G(f)"'] & H(A) \arrow[d, "H(f)"'] \\
	F(B) \arrow[r, "\phi(B)"'] & G(B) \arrow[r, "\eta(B)"'] & H(B)
	\end{tikzcd}
	\end{equation}
	
	 The meaning of \textit{vertical} will come to light when horizontal compositions are introduced in a bit.
\end{defn}
\begin{defn}[Functor categories]
	For any two categories $\mathbf{C}$ and $\mathbf{D}$, there is a \textbf{functor category}, denoted $\mathbf{D}^{\mathbf{C}}$ or $[\mathbf{C}, \mathbf{D}]$. Its objects are functors from $\mathbf{C}$ to $\mathbf{D}$, its morphisms are natural transformations between such functors and the composition is the one defined above. Associativity follows from associativity of composition in $\mathbf{D}$ and the identity morphism for a functor $F$ is $\one_F$.
\end{defn}
\begin{exmp}
	Recall that a left action of a group $G$ on a set $S$ is just a functor $\mathbf{B}G \rightsquigarrow \textbf{Set}$. Now, between two such functors $F,F' \in \textbf{Set}^{\mathbf{B}G}$, a natural transformation is a single map $\sigma: F(\ast) \rightarrow F'(\ast)$ such that $\sigma \circ F(g) = F'(g) \circ \sigma$ for any $g \in G$. In other words, denoting $\cdot$ for both the group action on $F(\ast)$ and on $F'(\ast)$, $\sigma$ satisfies $\sigma(g\cdot x) = g\cdot(\sigma(x))$ for any $g \in G$ and $x \in F(\ast)$. In group theory, such a map is called $G$-\textbf{equivariant}.
	
	Therefore, the category $\textbf{Set}^{\mathbf{B}A}$ can be identified as the category of $A$-sets (sets on which $A$ acts) with $A$-equivariant maps as the morphisms.
\end{exmp}
\begin{rem}
	Isomorphisms in a functor category are called \textbf{natural isomorphisms}, it is easy to show that they are natural transformations mapping every objects to isomorphisms. Functors that are naturally isomorphic are essentially the same functor; they send the same object to isomorphic objects and the same morphism to morphisms that are well-behaved under composition with isomorphisms between the source and targets.
\end{rem}

%%%%%%%%%%%%%%%Maybe not here.
%Before giving another example, we present a very nice result using limits. It is essentially saying that constructions inside the functor category $\mathbf{D}^{\mathbf{C}}$ are usually as simple as inside $\mathbf{C}$.
%\begin{prop}
%	Let $\mathbf{C}$, $\mathbf{D}$ and $J$ be categories. If all limits of shape $J$ exist in $\mathbf{C}$, then all such limits also exist in $\mathbf{D}^{\mathbf{C}}$.
%\end{prop}
%\begin{proof}
%	
%\end{proof}

It is now time to build intution for the horizontal composition of natural transformation which will ultimately lead to the notion of a 2-category.
\begin{defn}[The left action of functors]
	Let $F,F':\mathbf{C}\rightsquigarrow \mathbf{D}$, $G:\mathbf{D}\rightsquigarrow \mathbf{D}'$ be functors and $\phi:F\Rightarrow F'$ a natural transformation as summarized in diagram \eqref{diag-leftaction}.
	\begin{equation}\label{diag-leftaction}
	\begin{tikzcd}
	\mathbf{C} \arrow[rr, "F", bend left, squiggly] \arrow[rr, "F'"', bend right, squiggly] & \ \Big\Downarrow\phi & \mathbf{D} \arrow[r, "G", squiggly] & \mathbf{D}'
	\end{tikzcd}
	\end{equation}
	
	The functor $G$ acts on $\phi$ by sending it to $G\phi = A \mapsto G(\phi(A)) : \mathbf{C}_0 \rightarrow \mathbf{D}'_1$. Showing that diagram \eqref{diag-commleftaction} commutes for any $f \in \Hom_{\mathbf{C}}(A,B)$ will imply that $G\phi$ is a natural transformation from $G\circ F$ to $G\circ F'$ .
	\begin{equation}\label{diag-commleftaction}
		\begin{tikzcd}
		(G\circ F)(A) \arrow[d, "(G\circ F)(f)"'] \arrow[r, "G\phi(A)"] & (G\circ F')(A) \arrow[d, "(G\circ F')(f)"] \\
		(G\circ F)(B) \arrow[r, "G\phi(B)"] & (G\circ F')(B)
		\end{tikzcd}
	\end{equation}
	
	Consider this diagram after removing all applications of $G$, by naturality of $\phi$, it is commutative. Since functors preserve commuting diagrams, the diagram still commutes after applying $G$, hence $G\phi$ is indeed a natural transformation.
	
	It is also trivial to check that this constitutes a left action, namely, for any $G:\mathbf{D}\rightsquigarrow \mathbf{D}'$, $G':\mathbf{D}' \rightsquigarrow \mathbf{D}''$ and $\phi:F\Rightarrow F'$, \[\id_{\mathbf{D}}\phi = \phi \text{ and } G'(G\phi)= (G' \circ G)\phi.\]
\end{defn}

\begin{defn}[The right action of functors]
	Let $F,F':\mathbf{C}\rightsquigarrow \mathbf{D}$, $H:\mathbf{C}'\rightsquigarrow \mathbf{C}$ be functors and $\phi:F\Rightarrow F'$ a natural transformation as summarized in diagram \eqref{diag-rightaction}.
	\begin{equation}\label{diag-rightaction}
	\begin{tikzcd}
	\mathbf{C}' \arrow[r, "H", squiggly] & \mathbf{C} \arrow[rr, "F", bend left, squiggly] \arrow[rr, "F'"', bend right, squiggly] & \ \Big\Downarrow\phi & \mathbf{D}
	\end{tikzcd}
	\end{equation}
	
	The functor $H$ acts on $\phi$ by sending it to $\phi H = A \mapsto \phi(H(A)) : \mathbf{C}'_0 \rightarrow \mathbf{D}_1$. Showing that diagram \eqref{diag-commrightaction} commutes for any $f \in \Hom_{\mathbf{C}'}(A,B)$ will imply that $\phi H$ is a natural transformation from $F\circ H$ to $F'\circ H$.
	\begin{equation}\label{diag-commrightaction}
	\begin{tikzcd}
	(F\circ H)(A) \arrow[d, "(F\circ H)(f)"'] \arrow[r, "\phi H(A)"] & (F'\circ H)(A) \arrow[d, "(F'\circ H)(f)"] \\
	(F\circ H)(B) \arrow[r, "\phi H(B)"] & (F'\circ H)(B)
	\end{tikzcd}
	\end{equation}
	Commutativity of \eqref{diag-commrightaction} follows by naturality of $\phi$: change $f$ in diagram \eqref{diag-nattrans} with the morphism $H(f):H(A) \rightarrow H(B)$.
	
	It is also trivial to show this constitutes a right action, namely, for any $H:\mathbf{C}'\rightsquigarrow \mathbf{C}$, $H':\mathbf{C}''\rightsquigarrow \mathbf{C}'$ and $\phi:F\Rightarrow F'$,
	\[\phi \id_{\mathbf{C}} = \phi \text{ and } (\phi H)H' = \phi(H \circ H').\]
\end{defn}

\begin{prop}
	The two actions commute. Namely, in diagram \eqref{diag-commleftright}, $G(\phi H) = (G\phi) H$.
	\begin{equation}\label{diag-commleftright}
	\begin{tikzcd}
	\mathbf{C}' \arrow[r, "H", squiggly] & \mathbf{C} \arrow[rr, "F", bend left, squiggly] \arrow[rr, "F'"', bend right, squiggly] & \ \Big\Downarrow\phi & \mathbf{D} \arrow[r, "G", squiggly] & \mathbf{D}'
	\end{tikzcd}
	\end{equation}
\end{prop}
\begin{proof}
In both the L.H.S. and the R.H.S., an object $A \in \mathbf{C}_0$ is sent to $G(\phi(H(A)))$.
\end{proof}
We will refer to these two actions as the \textbf{biaction} of functors on natural transformations and they will motivate the definition of another way to compose natural transformations.

Let $\mathbf{C}$, $\mathbf{D}$ and $\mathbf{E}$ be categories, $H,H': \mathbf{C}\rightsquigarrow \mathbf{D}$ and $G,G':\mathbf{D} \rightsquigarrow \mathbf{E}$ be functors and $\phi:H\Rightarrow H'$ and $\eta:G\Rightarrow G'$ be natural transformations. These objects are summarized in diagram \eqref{diag-horizcompsetting}.
\begin{equation}\label{diag-horizcompsetting}
\begin{tikzcd}
\mathbf{C} \arrow[rr, "H", bend left, squiggly] \arrow[rr, "H'"', bend right, squiggly] & \ \Big\Downarrow \phi & \mathbf{D} \arrow[rr, "G", bend left, squiggly] \arrow[rr, "G'"', bend right, squiggly] & \ \Big\Downarrow \eta & \mathbf{E}
\end{tikzcd}
\end{equation}

The ultimate goal is to obtain a new composition of $\phi$ and $\eta$ that is a natural transformation $G\circ H \Rightarrow G'\circ H'$. Note that the biaction defined above yields four other natural transformation.
\begin{align*}
	G\phi&: G\circ H \Rightarrow G\circ H' &&\eta H: G\circ H \Rightarrow G'\circ H \\
	G'\phi&: G'\circ H \Rightarrow G'\circ H'&&\eta H': G\circ H' \Rightarrow G'\circ H'
\end{align*}
All of the functors involved go from $\mathbf{C}$ to $\mathbf{E}$, so all four natural transformations fit in diagram \eqref{diag-etdc} that lives in the category $[\mathbf{C},\mathbf{E}]$.
\begin{equation}\label{diag-etdc}
\begin{tikzcd}
G\circ H \arrow[r, "G\phi"] \arrow[d, "\eta H"'] & G\circ H' \arrow[d, "\eta H'"] \\
G'\circ H \arrow[r, "G'\phi"']                   & G'\circ H'                    
\end{tikzcd}
\end{equation}

At first glance, this suggests two different definitions for the horizontal composition, that is, the composition of the top path $(\eta H' \cdot G\phi)$ or the composition of the bottom path $(G'\phi \cdot \eta H)$. Surprisingly, both definitions coincide as shown in the next result.

\begin{lem}
	Diagram \eqref{diag-etdc} commutes.
\end{lem}
\begin{proof}
Fix an object $A \in \mathbf{C}_0$. Under $\eta H' \cdot G\phi$, it is sent to $\eta(H'(A)) \circ G(\phi(A))$ and under $G'\phi \cdot \eta H$, it is sent to $G'(\phi(A)) \circ \eta(H(A))$. Thus, the proposition is equivalent to saying diagram \eqref{diag-proofhorizcomp} is commutative (in $\mathbf{E}$).
\begin{equation}\label{diag-proofhorizcomp}
\begin{tikzcd}
(G\circ H)(A) \arrow[r, "G(\phi(A))"] \arrow[d, "\eta(H(A))"'] & (G\circ H')(A) \arrow[d, "\eta(H'(A))"] \\
(G'\circ H)(A) \arrow[r, "G'(\phi(A))"']                       & (G'\circ H')(A)                        
\end{tikzcd}
\end{equation}
This follows from the naturality of $\eta$ (in diagram \eqref{diag-nattrans}, replace $A$ with $H(A)$, $B$ with $H'(A)$, $f$ with $\phi(A)$, $F$ with $G$ and $G$ with $G'$).
\end{proof}

\begin{defn}[Horizontal composition]\label{horizcomp}
	In the setting described in \eqref{diag-horizcompsetting}, we define the \textbf{horizontal composition} of $\eta$ and $\phi$ by $\eta \diamond \phi = \eta H' \cdot G\phi = G'\phi\cdot \eta H$.
\end{defn}
As is expected from the terminology, the composition $\diamond$ is associative.
\begin{prop}
	In the setting of diagram \eqref{diag-assochorizcomp}, $\psi \diamond (\eta \diamond \phi)= (\psi \diamond \eta)\diamond \phi$.
	\begin{equation}\label{diag-assochorizcomp}
	\begin{tikzcd}
	\mathbf{C} \arrow[rr, "H", bend left, squiggly] \arrow[rr, "H'"', bend right, squiggly] & \ \Big\Downarrow \phi & \mathbf{D} \arrow[rr, "G", bend left, squiggly] \arrow[rr, "G'"', bend right, squiggly] & \ \Big\Downarrow \eta & \mathbf{E} \arrow[rr, "K", bend left, squiggly] \arrow[rr, "K'"', bend right, squiggly] & \ \Big \Downarrow \psi & \mathbf{F}
	\end{tikzcd}
	\end{equation}
\end{prop}
\begin{proof}
	Similarly to how we constructed the diagram in $[\mathbf{C},\mathbf{E}]$ previously, we can use the biaction of functors on natural transformations and composition of functors to obtain the following diagram in $[\mathbf{C}, \mathbf{F}]$ (the $\circ$'s are left out for simplicity).
	\begin{equation}
	\begin{tikzcd}
	& K'GH \arrow[dd, "K'\eta H" near start] \arrow[rr, "K'G\phi"] &                                                    & K'GH' \arrow[dd, "K'\eta H'"] \\
	KGH \arrow[rr, crossing over, "KG\phi"' near end] \arrow[dd, "K\eta H"'] \arrow[ru, "\psi GH"] &                                                    & KGH'  \arrow[ru, "\psi GH'"'] &                               \\
	& K'G'H \arrow[rr, "K'G'\phi" near start]                       &                                                    & K'G'H'                        \\
	KG'H \arrow[rr, "KG'\phi"'] \arrow[ru, "\psi G'H"]                    &                                                    & KG'H'\arrow[uu, <-, crossing over, "K\eta H'" near start] \arrow[ru, "\psi G'H'"']                     &                              
	\end{tikzcd}
	\end{equation}
	
	This commutes because each face of the cube corresponds to a variant of diagram \eqref{diag-etdc} (with some substitutions and application of a functor) and combining commutative diagrams yields commutative diagrams. Then, it follows easily that $\diamond$ is associative.
\end{proof}

There is one last thing to conclude that \textbf{Cat} is a 2-category, namely, that the vertical and horizontal compositions interact nicely.
\begin{prop}[Interchange identity]
	In the setting of \eqref{diag-interidsetting}, the \textbf{interchange identity} holds:
	\begin{equation}\label{eqn-interid}\tag{$*$}
		(\eta' \cdot \eta) \diamond (\phi' \cdot \phi) = (\eta' \diamond \phi') \cdot (\eta \diamond \phi).
	\end{equation}
	
	\begin{equation}\label{diag-interidsetting}
	\begin{tikzcd}
	\mathbf{C} \arrow[rr, "H", bend left=49, squiggly] \arrow[rr, "H''"', bend right=49, squiggly] \arrow[rr, "H'" near end, squiggly] & \begin{matrix}\big\Downarrow \phi\\ \\\big\Downarrow \phi'\end{matrix} & \mathbf{D} \arrow[rr, "G", bend left=49, squiggly] \arrow[rr, "G''"', bend right=49, squiggly] \arrow[rr, "G'" near end, squiggly] & \begin{matrix}\big\Downarrow \eta\\ \\\big\Downarrow \eta'\end{matrix} & \mathbf{E}
	\end{tikzcd}
	\end{equation}
\end{prop}
\begin{proof}
	Again, this proof is just a matter of combining the right diagrams. After combining the diagrams in $[\mathbf{C},\mathbf{E}]$ corresponding to $\eta \diamond \phi$ and $\eta'\diamond \phi'$, it is easy to see that the R.H.S. of \eqref{eqn-interid} is the morphism going from $G\circ H$ to $G''\circ H''$ (see \eqref{diag-rhsinterid}).
	\begin{equation}\label{diag-rhsinterid}
	\begin{tikzcd}
	G\circ H \arrow[r, "G\phi"] \arrow[d, "\eta H"'] & G\circ H' \arrow[d, "\eta H'"]                       &                                  \\
	G'\circ H \arrow[r, "G'\phi"']                   & G'\circ H' \arrow[d, "\eta'H'"'] \arrow[r, "G'\phi'"] & G'\circ H'' \arrow[d, "\eta'H''"] \\
	& G''\circ H' \arrow[r, "G''\phi'"']                   & G''\circ H''                    
	\end{tikzcd}
	\end{equation}

	Moreover, observe that the diagram corresponding to the L.H.S. can be factored with the following equations.
	\begin{align*}
	(\eta'\cdot \eta)H = \eta'H\cdot \eta H && (\eta'\cdot \eta)H'' = \eta'H''\cdot \eta H''\\
	G(\phi'\cdot \phi) = G\phi'\cdot G\phi && G''(\phi'\cdot \phi) = G''\phi'\cdot G''\phi
	\end{align*}
	Combining the factored diagram with \eqref{diag-rhsinterid}, we obtain \eqref{diag-interid} from which the interchange identity follows immediately.
	\begin{equation}\label{diag-interid}
	\begin{tikzcd}
	G\circ H \arrow[r, "G\phi"] \arrow[d, "\eta H"']    & G\circ H' \arrow[d, "\eta H'"] \arrow[r, "G\phi'"]    & G\circ H'' \arrow[d, "\eta H''"]  \\
	G'\circ H \arrow[r, "G'\phi"'] \arrow[d, "\eta'H"'] & G'\circ H' \arrow[d, "\eta'H'"'] \arrow[r, "G'\phi'"] & G'\circ H'' \arrow[d, "\eta'H''"] \\
	G''\circ H \arrow[r, "G''\phi"']                     & G''\circ H' \arrow[r, "G''\phi'"']                    & G''\circ H''                     
	\end{tikzcd}
	\end{equation}
\end{proof}

\begin{defn}[Strict 2-cateory]
	A \textbf{strict 2-category} consists of
	\begin{itemize}
		\item a category $\mathbf{C}$,
		\item for every $A,B \in \mathbf{C}_0$ a category $\mathbf{C}(A,B)$ with $\Hom_{\mathbf{C}}(A,B)$ as its objects (composition is denoted $\cdot$ and identities $\one$) and morphisms are called 2-morphisms,
		\item a category with $\mathbf{C}_0$ as its objects, where the morphisms are pairs of parallel morphisms of $\mathbf{C}$ along with a 2-morphism between them (called a \textbf{2-cell}) and the identity map sends $A \in \mathbf{C}_0$ to the pair $(\id_A, \id_A)$ and the 2-morphism $\one_{\id_A}$ (composition is denoted $\diamond$),
	\end{itemize} 
	such that the interchange identity holds \eqref{eqn-interid}.
\end{defn}
We will probably not cover it in the first season but there are notions of morphisms between 2-categories called 2-functors, between 3-categories as well as between n-categories for any $n$ (even $n= \infty$!), these objects are more deeply studied in higher category theory.

\section{Equivalences}
As is expected, an isomorphism of categories is an isomorphism in the category \textbf{Cat}, namely, a functor $F:\mathbf{C}\rightsquigarrow \mathbf{D}$ with an inverse $G:\mathbf{D}\rightsquigarrow \mathbf{C}$ such that $F \circ G = \id_{\mathbf{D}}$ and $G\circ F = \id_{\mathbf{C}}$. As is typical in mathematics, one cannot distinguish between isomorphic categories as they only differ in notations and terminology.
\begin{exmps}
	\begin{enumerate}
		\item[]
		\item It was already shown that for a group $A$, the category $\textbf{Set}^{\mathbf{B}A}$ is isomorphic to the category of $A$-sets with equivariant maps as morphisms.
		\item The category \textbf{Rel} of sets with relations is isomorphic to $\op{\textbf{Rel}}$.
		\item Let $k$ be a field and $G$ a finite group, the categories of $k[G]$-modules and of $k$-linear representations of $G$ are isomorphic when $G$ is finite.
	\end{enumerate}
\end{exmps}
 Although there are other interesting instances of isomorphic categories, natural transformations lead to a more nuanced equality between two categories, that is, equivalence.
\begin{defn}[Equivalence]
	A functor $F:\mathbf{C}\rightsquigarrow \mathbf{D}$ is an \textbf{equivalence} of categories if there exists a functor $G:\mathbf{D}\rightsquigarrow \mathbf{C}$ such that $F\circ G\cong \id_{C}$ and $G\circ F \cong \id_{D}$, where $\cong$ denotes natural isomorphism. Two categories $\mathbf{C}$ and $\mathbf{D}$ are \textbf{equivalent}, denoted $\mathbf{C} \simeq \mathbf{D}$, if there is an equivalence between them.
\end{defn}
In order to gain more intuition on how equivalences equate two categories, let us observe what properties this forces on the functor $F$.

For any morphisms $f \in \Hom_{\mathbf{C}}(A,B)$, the following square commutes where $\phi(A)$ and $\phi(B)$ are isomorphisms.
\begin{equation}\label{diag-bijfromequiv}
\begin{tikzcd}
A \arrow[r, "f"] \arrow[d]                                 & B \arrow[d]                            \\
GF(A) \arrow[r, "GF(f)"'] \arrow[u, "\phi(A)"] & GF(B) \arrow[u, "\phi(B)"']
\end{tikzcd}
\end{equation}

This implies that the map $f \mapsto GF(f):\Hom_{\mathbf{C}}(A,B) \rightarrow \Hom_{\mathbf{C}}(GF(A), GF(B))$ is a bijection. Indeed, pre-composition by $\phi(A)^{-1}$ and post-composition by $\phi(B)$ are both bijections (recall the definitions of monics and epics), so \[f \mapsto \phi(B) \circ f \circ \phi(A)^{-1} = GF(f)\]is a bijection. Since $A$ and $B$ are arbitrary, $G\circ F$ is a fully faithful functor and a symmetric argument shows $F\circ G$ is also fully faithful. Then, it is easy to conclude that $F$ and $G$ must be fully faithful as well.

What is more, the existence of an isomorphism $\eta(A): A \rightarrow FG(A)$ for any object $A$ implies $F$ (symmetrically $G$) has the following property.
\begin{defn}[Essentially surjective]
	A functor $F:\mathbf{C}\rightsquigarrow \mathbf{D}$ is \textbf{essentially surjective} if for any $X \in \mathbf{D}_0$, there exists $Y \in \mathbf{C}_0$ such that $X \cong F(Y)$.
\end{defn}
We will show that these two properties are necessary and sufficient for $F$ being an equivalence.
\begin{thm}
	A functor $F:\mathbf{C}\rightsquigarrow \mathbf{D}$ is an equivalence of categories if and only if $F$ is fully faithful and essentially surjective.
\end{thm}
\begin{proof}
	($\Rightarrow$) Shown above.
	
	($\Leftarrow$) We construct a functor $G:\mathbf{D}\rightsquigarrow \mathbf{C}$ such that $G\circ F \cong \id_{\mathbf{C}}$ and $F\circ G \cong \id_{\mathbf{D}}$. Since $F$ is essentially surjective, for any $A \in \mathbf{D}_0$, there exists an object $G(A) \in \mathbf{C}_0$ and an isomorphism $\phi(A):F(G(A)) \cong A$. Hence, $A \mapsto G(A)$ is a good candidate to describe the action of $G$ on objects.
	
	Next, similarly to the converse direction, note that for any $A,B \in \mathbf{D}_0$, the map 
	\[f\mapsto \phi(B) \circ f \circ \phi(A)^{-1}\]
	is a bijection from $\Hom_{\mathbf{D}}(A,B)$ to $\Hom_{\mathbf{D}}(FG(A), FG(B))$. Moreover, since the functor $F$ is fully faithfull, it induces a bijection $F_1: \Hom_{\mathbf{C}}(G(A), G(B)) \rightarrow \Hom_{\mathbf{D}}(FG(A), FG(B))$ which in turns yields a bijection 
	\[G: \Hom_{\mathbf{D}}(A,B) \rightarrow \Hom_{\mathbf{C}}(G(A), G(B)) = f \mapsto F_1^{-1}(\phi(B) \circ f \circ \phi(A)^{-1}).\]
	This is the action of $G$ on morphisms. Observe that the construction of $G$ ensures that $F\circ G \cong \id_{\mathbf{D}}$ through the natural transformation $\phi$. It remains to show that $G$ is indeed a functor and find a natural isomorphism $\eta:G\circ F \cong \id_{\mathbf{C}}$.
	
	For any composable morphisms $(f,g)$, it is easy to verify that 
	\[F(G(f)\circ G(g)) = FG(f) \circ FG(g) = FG(f \circ g),\]
	so functoriality of $G$ follows after applying $F_1^{-1}$. To find $\eta$, recall that the definition of $G$ yields diagram \eqref{diag-findingeta} for any $f\in \Hom_{\mathbf{C}}(A,B)$.
	\begin{equation}\label{diag-findingeta}
		\begin{tikzcd}
	F(A) \arrow[d] \arrow[r, "F(f)"]                    & F(B) \arrow[d]                  \\
	FGF(A) \arrow[u, "\phi(F(A))"] \arrow[r, "FGF(f)"'] & FGF(B) \arrow[u, "\phi(F(B))"']
	\end{tikzcd}
	\end{equation}
	
	Then, because $F$ is fully faithful, the following square also commutes in $\mathbf{C}$ where $\eta = X \mapsto F_1^{-1}(\phi(F(X)))$ and we conclude that $\eta$ is a natural isomorphism $\id_{\mathbf{C}} \cong G\circ F$.
	\begin{equation}\label{diag-foundeta}
	\begin{tikzcd}
	A \arrow[d] \arrow[r, "f"]                     & B \arrow[d]                 \\
	GF(A) \arrow[u, "\eta(A)"] \arrow[r, "GF(f)"'] & GF(B) \arrow[u, "\eta(B)"']
	\end{tikzcd}
	\end{equation}
\end{proof}
The insight to extract from this argument is that two categories are equivalent if they describe the same objects and morphisms with the only relaxation that isomorphic objects can appear any number of times in either category. In contrast, categories can only be isomorphic if they have exactly the same objects and morphisms.

\begin{rem} We used the axiom of choice to construct the quasi-inverse of $F$.
\end{rem}

\begin{exmps} Examples of significant equivalences are all over the place in higher mathematics. However, they require a bit of work to describe them, thus let us only say a few words on them.
	\begin{enumerate}
		\item An early result in linear algebra says that any finite dimensional vector space over a field $k$ is isomorphic to $k^n$ for some $n\in \N$. Thus, it is easy to see that the category whose objects are $k^n$ for all $n\in \N$ and morphisms are $m\times n$ matrices with entries in $k$ is equivalent to the category of finite dimensional vector spaces.
		\item The equivalence between the category of affine scheme and the opposite of the category of commutative rings is a seminal result in algebraic geometry, in particular it is the advent of scheme theory.
		\item The equivalence between Boolean lattices and Stone spaces is again seminal in the theory of Stone-type dualities. These can lead to deep connections between topology and logic. One application in particular is the study of the behavior of computer programs through formal semantics.
	\end{enumerate}
\end{exmps}
\end{document}

