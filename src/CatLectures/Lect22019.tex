\documentclass{article}

\newcommand{\bra}[1]{\left(#1\right)}
\usepackage[activate={true,nocompatibility},final,tracking=true,kerning=true,spacing=true,factor=1100,stretch=10,shrink=10]{microtype}
\microtypecontext{spacing=nonfrench}
\usepackage{tikz}
\usepackage{tikz-cd}
\usepackage{mathpazo}
\usepackage{amsmath,amsthm,amssymb}
\usepackage{subcaption}
\usepackage{enumerate}
\usetikzlibrary{shapes}
\usetikzlibrary{positioning}
% Set up the images/graphics package
\usepackage{graphicx,float}
\setkeys{Gin}{width=\linewidth,totalheight=\textheight,keepaspectratio}
\graphicspath{{.}}


% Small sections of multiple columns
\usepackage{multicol}
\usepackage[margin=1.6in]{geometry}

%--------Theorem Environments--------
%theoremstyle{plain} --- default
\newtheorem{thm}{Theorem}
\newtheorem{cor}[thm]{Corollary}
\newtheorem{prop}[thm]{Proposition}
\newtheorem{lem}[thm]{Lemma}
\newtheorem{fact}[thm]{Fact}
\newtheorem{conj}[thm]{Conjecture}
\newtheorem{quest}[thm]{Question}
\newtheorem{claim}{Claim}

\theoremstyle{definition}
\newtheorem{defn}[thm]{Definition}
\newtheorem{defns}[thm]{Definitions}
\newtheorem{con}[thm]{Construction}
\newtheorem{exmp}[thm]{Example}
\newtheorem{jk}[thm]{Joke}
\newtheorem{exmps}[thm]{Examples}
\newtheorem{notn}[thm]{Notation}
\newtheorem{notns}[thm]{Notations}
\newtheorem{addm}[thm]{Addendum}
\newtheorem{exer}[thm]{Exercise}

\theoremstyle{remark}
\newtheorem{rem}[thm]{Remark}
\newtheorem{ans}[thm]{Answer}
\newtheorem{rems}[thm]{Remarks}
\newtheorem{warn}[thm]{Warning}
\newtheorem{sch}[thm]{Scholium}

% MACROS
\newcommand{\Mod}[1]{\ (\text{mod}\ #1)}
\newcommand{\R}{\mathbb{R}}
\newcommand{\N}{\mathbb{N}}
\newcommand{\Q}{\mathbb{Q}}
\newcommand{\F}{\mathbb{F}}
\newcommand{\Z}{\mathbb{Z}}
\newcommand{\mC}{\mathcal{C}}
\newcommand{\mG}{\mathcal{G}}
\newcommand{\mP}{\mathcal{P}}
\newcommand{\one}{\mathbb{1}}
\renewcommand{\P}{\mathbb{P}}
\DeclareMathOperator{\dist}{dist}
\DeclareMathOperator{\aut}{Aut}
\DeclareMathOperator{\gal}{Gal}
\DeclareMathOperator{\orb}{Orb}
\DeclareMathOperator{\stab}{Stab}
\DeclareMathOperator{\inn}{Inn}
\DeclareMathOperator{\spn}{Span}
\DeclareMathOperator{\out}{Out}
\DeclareMathOperator{\im}{Im}
\DeclareMathOperator{\arr}{Arr}
\DeclareMathOperator{\rk}{rk}
\DeclareMathOperator{\rcf}{rcf}
\DeclareMathOperator{\tors}{Tors}
\DeclareMathOperator{\Hom}{Hom}
\DeclareMathOperator{\ann}{Ann}
\DeclareMathOperator{\syl}{Syl}
\newcommand{\norm}[1]{\left\lVert #1 \right\rVert}
\newcommand{\inp}[2]{\left\langle #1, #2 \right\rangle}
\newcommand{\id}{\text{id}}
\newcommand{\gln}{\text{GL}_n}
\newcommand{\op}[1]{#1^{\text{op}}}

\title{Lecture 2 - Duality and More Vocabulary\vspace{-10pt}}
\author{Ralph Sarkis}
\date{\vspace{-10pt}May 15, 2019\vspace{-15pt}}  % if the \date{} command is left out, the current date will be used
\begin{document}
\maketitle
\begin{abstract} Through the exploration of duality and the presentation of more vocabulary, further familiarity with categories and functors will be built.
\end{abstract}
\section{Duality}
The concept of duality is ubiquitous throughout mathematics. It can relate two perspectives of the same object as for dual vector spaces, two complementary problems such as a maximization and a minimization linear program and even two seemingly unrelated fields like topology and logic (Stone dualities). While this vague principle of duality is the foundation of many groundbreaking results, the duality in question here is categorical duality and it is a bit more precise.

Informally, there is nothing more to say than ``Take all the diagrams in a definition/theorem, reverse the arrows and reap the benefits of the dual concept/result.`` The more formal version will follow after we first exhibit the principle in action.

Recall that, intuitively, a functor is a structure preserving transformation between categories. A simple example we gave last lecture was functors between posets that were order-preserving functions. However, as a consequence, one might conclude that order-reversing functions impair the structure of a poset, which feels arbitrary. The same happens between categories representing groups because anti-homomorphisms cannot arise as functors between such categories.

There are two options to remedy this discrepancy between intuition and formalism; both have duality as a guiding principle.

\subsection{Contravariant functors}
By modifying last week's definition to require that $F(f)$ goes in the opposite direction, we obtain a \textbf{contravariant} functor. Incidentally, what we defined as a functor last week is more commonly called a \textbf{covariant} functor.
\begin{defn}[Contravariant functor]
	Let $C$ and $D$ be categories, a \textbf{contravariant functor} $F: C \rightsquigarrow D$ is a pair of maps $F_0:C_0 \rightarrow D_0$ and $F_1:C_1 \rightarrow D_1$ such that the following diagrams commute (where $F_2$ is now induced by the definition of $F_1$ with $(f,g) \mapsto (F_1(g), F_1(f))$).
	\begin{figure}[h]
		\centering
		\begin{tikzcd}
			C_0 \arrow[d, "F_0"'] & C_1 \arrow[d, "F_1"] \arrow[l, "s"'] \arrow[r, "t"] & C_0 \arrow[d, "F_0"] \\
			D_0 & D_1 \arrow[l, "t"] \arrow[r, "s"'] & D_0
		\end{tikzcd}
		\qquad 
		\begin{tikzcd}
			C_2 \arrow[d, "\circ_C"'] \arrow[r, "F_2"] & D_2 \arrow[d, "\circ_D"] \\
			C_1 \arrow[r, "F_1"'] & D_1
		\end{tikzcd}
		\qquad
		\begin{tikzcd}
			C_0 \arrow[d, "u_C"'] \arrow[r, "F_0"] & D_0 \arrow[d, "u_D"] \\
			C_1 \arrow[r, "F_1"'] & D_1
		\end{tikzcd}
	\end{figure}
	\newpage
	In words, $F$ must satisfy the following properties.
	\begin{enumerate}[i.]
		\item For any $A, B \in C_0$, if $f \in \Hom_C(A,B)$ then $F(f) \in \Hom_D(F(B), F(A))$.
		\item If $f,g \in C_1$ are composable, then $F(f\circ g) = F(g) \circ F(f)$.
		\item If $A \in C_0$, then $u_D(F(A)) = F(u_C(A))$.
	\end{enumerate}
\end{defn}
\begin{exmps}
	Just like their covariant counterparts, contravariant functors are quite numerous. Here are a few simple ones, we leave you to check that they satisfy the diagrams above.
	\begin{enumerate}
		\item It is easy to verify that contravariant functors $F: (X, \leq) \rightsquigarrow (Y, \subseteq)$ correspond to order-reversing functions between the posets while contravariant functors $F: G\ast \rightsquigarrow H\ast$ correspond to anti-homomorphisms between the groups.
		\item The contravariant powerset functor $\widehat{\mP}: \textbf{Set} \rightsquigarrow \textbf{Set}$ sends a set $X$ to its power set $\mP(X)$ and a function $f: X\rightarrow Y$ to the pre-image map $\mP(f):\mP(Y)\rightarrow \mP(X)$, the latter sends a subset $S\subseteq Y$ to \[f^{-1}(S) = \{x \in X \mid f(x) \in S\} \subseteq X.\]
	\end{enumerate}
\end{exmps}
Next, there is a couple of functors that are key to understand the philosophy put forward by category theory (we will talk more about it when covering the Yoneda lemma).
\begin{exmp}[Hom functors]
	Let $C$ be a locally-small category and $A \in C_0$ one of its object. We define the covariant and contravariant $\Hom$ functors from $C$ to $\textbf{Set}$.
	\begin{enumerate}
		\item The functor $\Hom_C(A,-): C \rightsquigarrow \textbf{Set}$ sends an object $B\in C_0$ to the Hom-set $\Hom_C(A,B)$ and a morphism $f:B\rightarrow B'$ to the function $$\Hom_C(A,f): \Hom_C(A,B) \rightarrow \Hom_C(A,B') = g \mapsto f\circ g.$$
		This function is called post-composition by $f$ and is sometimes denoted $f \circ (-)$. Let us show $\Hom_C(A, -)$ is a covariant functor.
		\begin{enumerate}[i.]
			\item For any $f \in C_1$, it is clear from the definitions that \[\Hom_C(A,s(f)) = s(\Hom_C(A,f)) \text{ and } \Hom_C(A,t(f)) = t(\Hom_C(A,f)).\]
			\item For any $(f_1,f_2) \in C_2$, we claim that \[\Hom_C(A,f_1\circ f_2) = \Hom_C(A,f_1)\circ \Hom_C(A,f_2).\] In the L.H.S., an element $g \in \Hom_C(A,s(f_1\circ f_2))$ is mapped to $(f_1 \circ f_2) \circ g$ and in the R.H.S., an element $g \in \Hom_C(A,s(f_2)$ is mapped to $f_1\circ (f_2 \circ g)$. Since $s(f_1 \circ f_2) = s(f_2)$ and composition is associative, we conclude that the two maps are the same.
			\item For any $A \in C_0$, the post-composition by $u_C(A)$ is defined to be the identity, hence the third diagram also commutes.
		\end{enumerate}
		\item The functor $\Hom_C(-,A): C \rightsquigarrow \textbf{Set}$ sends an object $B\in C_0$ to the Hom-set $\Hom_C(B,A)$ and a morphism $f:B\rightarrow B'$ to the function $$\Hom_C(f,A): \Hom_C(B',A) \rightarrow \Hom_C(B,A) = g \mapsto g\circ f.$$
		This function is called pre-composition by $f$ and is sometimes denoted $(-) \circ f$. Let us show $\Hom_C(-,A)$ is a contravariant functor.
		\begin{enumerate}[i.]
			\item For any $f \in C_1$, it is clear from the definitions that \[\Hom_C(s(f),A) = t(\Hom_C(f,A))\text{ and } \Hom_C(t(f),A) = s(\Hom_C(f,A)).\]
			\item For any $(f_1,f_2) \in C_2$, we claim that \[\Hom_C(f_1\circ f_2,A) = \Hom_C(f_2,A)\circ \Hom_C(f_1,A).\] In the L.H.S., an element $g \in \Hom_C(t(f_1\circ f_2),A)$ is mapped to $g\circ (f_1 \circ f_2)$ and in the R.H.S., an element $g \in \Hom_C(t(f_1),A)$ is mapped to $(g\circ f_1) \circ f_2$. Since $t(f_1 \circ f_2) = t(f_1)$ and composition is associative, we conclude that the two maps are the same.
			\item For any $A \in C_0$, the pre-composition by $u_C(A)$ is defined to be the identity, hence the third diagram also commutes.
		\end{enumerate}
	\end{enumerate}
\end{exmp}
\subsection{Opposite Category}
Another way to deal with order-reversing maps $(X, \leq) \rightarrow (Y, \subseteq)$ is to consider the reverse order on $X$ and a covariant functor $(X, \geq) \rightsquigarrow (Y, \subseteq)$. This also works for anti-homomorphims by constructing the opposite group $\op{G}$ in which the operation is reversed, namely $g\op{\cdot} h = hg$. The opposite category is a generalization of these constructions.

\begin{defn}[Opposite category]
	Let $C$ be a category, we denote the \textbf{opposite category} with $\op{C}$ and define it by 
	\[ \op{C}_0 = C_0,\ \op{C}_1 = C_1,\ \op{s} = t,\ \op{t} = s,\ u_{\op{C}} = u_C\]
	with the composition defined by $\op{f}\op{\circ}\op{g} = \op{(g\circ f)}$. This canonically leads to the following contravariant functor $\op{(-)}_C: C \rightsquigarrow \op{C}$ which sends an object $A$ to $\op{A}$ and a morphism $f$ to $\op{f}$. It is called the \textbf{opposite functor}.
\end{defn}
\begin{rem}
	This definition leads to seeing contravariant functors as covariant functors. Formally, let $F:C\rightsquigarrow D$ be a contravariant functor, we can view $F$ as covariant functor from $\op{C}$ to $D$ or from $C$ to $\op{D}$ via the compositions $F\circ \op{(-)}_{\op{C}}$ and $\op{(-)}_{D}\circ F$ respectively. Note that the $\op{}$ notation here is just used to distinguish elements in $C$ and $\op{C}$ but the class of objects and morphisms are the same.
\end{rem}

\begin{exmps}
	As hinted at before, the category corresponding to $(X, \geq)$ is the opposite category of $(X, \leq)$ and $\op{(G\ast)}$ is the category corresponding to the opposite group of $G$. While there are other interesting examples, the opposite construction is usually used implicitly to avoid dealing with contravariant functors or to avoid proving the dual of an already proven result.
\end{exmps}

Let us continue to illustrate how duality can be useful with some simple definitions and results.
\begin{defn}[Monomorphism]
	Let $C$ be a category, a morphism $f \in C_1$ is said to be \textbf{monic} (or a \textbf{monomorphism}) if for any $(f,g), (f,h) \in C_2$ where $g$ and $h$ have the same source, $f\circ g = f\circ h$ implies $g = h$. Equivalently, $f$ is monic if $g = h$ whenever the following diagram commutes.
	\begin{figure}[h]
		\centering
		\begin{tikzcd}
			\bullet \arrow[r, "h"', bend right] \arrow[r, "g", bend left] & \bullet \arrow[r, "f"] & \bullet
		\end{tikzcd}
	\end{figure}
\end{defn}
\begin{prop}\label{propmon1}
	Let $C$ be a category and $f:A\rightarrow B$ a morphism, if there exists $f': B\rightarrow A$ such that $f'\circ f = \id_A$, then $f$ is a monomorphism.
\end{prop}
\begin{proof}
	If $f\circ g = f\circ h$, then $f'\circ f \circ g = f'\circ f \circ h$ implying $g = h$.
\end{proof}
\begin{prop}\label{propmon2}
	Let $C$ be a category and $(f_1, f_2) \in C_2$, if $f_1 \circ f_2$ is a monomorphism, then $f_2$ is a monomorphism.
\end{prop}
\begin{proof}
	Let $g,h \in C_1$ be such that $f_2\circ g = f_2\circ h$, we immediately get that $(f_1\circ f_2)\circ g = (f_1 \circ f_2) \circ h$. Since $f_1\circ f_2$ is a monomorphism, this implies $g = h$.
\end{proof}
It is now obvious that monomorphisms are analogous to injective functions and we will see that they are exactly the same in the category \textbf{Set}, but first let us introduce the dual concept.
\begin{defn}[Epimorphism]
	Let $C$ be a category, a morphism $f \in C_1$ is said to be \textbf{epic} (or an \textbf{epimorphism}) if for any two morphisms $(g,f), (h,g) \in C_2$ where $g$ and $h$ have the same target, $g\circ f = h\circ f$ implies $g = h$. Equivalently, $f$ is epic if $g = h$ whenever the following diagram commutes.
	\begin{figure}[h]
		\centering
		\begin{tikzcd}
			\bullet \arrow[r, "f"] & \bullet \arrow[r, "g", bend left] \arrow[r, "h"', bend right] & \bullet
		\end{tikzcd}
	\end{figure}
\end{defn}
The duals to Propositions \ref{propmon1} and \ref{propmon2} also hold. Although doing the straightforward proofs is very easy, the two next proofs rely on duality and convey the general sketch that works anytime a dual result needs to be proven.
\begin{prop}\label{propep1}
	Let $C$ be a category and $f:A\rightarrow B$ a morphism, if there exists $f': B\rightarrow A$ such that $f\circ f' = \id_B$, then $f$ is epic.
\end{prop}
\begin{proof}
	Observe that $f$ is epic in $C$ if and only if $\op{f}$ is monic in $\op{C}$ (reverse the arrows in the definition). Moreover, by definition, \[\op{f'} \circ \op{f} = \op{(f \circ f')} = \op{\id_B} = \id_{\op{B}},\] so by the result for monomorphisms, $\op{f}$ is monic and hence $f$ is epic. 
\end{proof}
\begin{prop}
	Let $C$ be a category and $(f_1, f_2) \in C_2$, if $f_1 \circ f_2$ is epic, then $f_2$ is epic.
\end{prop}
\begin{proof}
	Since $\op{f_2} \circ \op{f_1} = \op{(f_1 \circ f_2)}$ is monic, the result for monomorphisms implies $\op{f_2}$ is monic and hence $f_2$ is epic.
\end{proof}
\begin{exmp}[\textbf{Set}]
	\begin{itemize}
		\item[]
		\item A function $f:A\rightarrow B$ is a monomorphism if and only if it is injective:
		
		($\Leftarrow$) Since $f$ is injective, it has a left inverse, so it is epic by Proposition \ref{propmon1}.
		
		($\Rightarrow$) For any $a \in A$, let $g_a: \{1\} \rightarrow A$ be the function sending $1$ to $a$. Fix $a_1, a_2 \in A$, the functions $g_{a_1}$ and $g_{a_2}$ are different, hence $f \circ g_{a_1} \neq f \circ g_{a_2}$. Therefore, $f(a_1) \neq f(a_2)$ and since $a_1$ and $a_2$ were arbitrary, $f$ is injective.
		
		\item A function $f:A\rightarrow B$ is an epimorphism if and only if it is surjective:
		
		($\Leftarrow$) Since $f$ is surjective, it has a right inverse, so it is epic by Proposition \ref{propep1}.
		
		($\Rightarrow$) Let $h$ be the constant function at $1$ and $g$ be the indicator function of $\im(f) \subseteq B$, namely, \[g(x) = \begin{cases}1&\exists a \in A, x = f(a)\\0&o/w\end{cases}.\]
		It is clear that $g \circ f = h\circ f$ and since $f$ is epic, it implies $g = h$. Thus, any element of $B$ is in the image of $f$, that is $f$ is surjective.
	\end{itemize}
\end{exmp}

\begin{exmp}[\textbf{Mon}]
Inside the category \textbf{Mon} where objects are monoids and morphims are monoid homomorphisms, the monomorphisms correspond exactly to injective homomorphims.

($\Rightarrow$) Let $f:M\rightarrow M'$ be an injective homomorphims and $g_1,g_2:N\rightarrow M$ be two parallel homomorphisms. Suppose that $f\circ g_1 = f\circ g_2$, then for all $x \in N$, $f(g_1(x)) = f(g_2(x))$, so by injectivity of $f$, $g_1(x) = g_2(x)$. Therefore $g_1 = g_2$ and since $g_1$ and $g_2$ were arbitrary, $f$ is a monomorphism.

($\Leftarrow$) Let $f:M\rightarrow M'$ be a monomorphism. Let $x,y \in M$ and define $p_x :\N \rightarrow M$ by $k\mapsto x^k$ and similarly for $p_y$. It is trivial to show that $p_x$ and $p_y$ are homomorphism. If $f(x) = f(y)$, then, by the homomorphism property, for all $k \in \N$
\[f(p_x(k))= f(x^k) = f(x)^k  = f(y)^k = f(y^k) = f(p_y(k)).\]
In other words, we get $f\circ p_x = f \circ p_y$, so $p_x = p_y$ and $x = y$. We conclude that $f$ is injective.

Conversely, an epimorphism is not necessarily surjective. For example, the inclusion homomorphism $i:\N \rightarrow \Z$ is clearly not surjective but it is an epimorphism. Indeed, let $g,h: \Z\rightarrow M$ be two monoid homomorphisms satisfying $g \circ i = h\circ i$. In particular, $g(n) = h(n)$ for any $n \in \N\subset \Z$. It remains to show that also $g(-n) = h(-n)$, but if it were not the case for some $n$, $g(n)$ would have two left inverses $g(-n)$ and $h(-n)$ which is not possible. Hence, $g = h$ and $i$ is an epimorphism.
\end{exmp}
\begin{defn}[Isomorphism]
	Let $C$ be a category, a morphism $f:A\rightarrow B$ is said to be an \textbf{isomorphism} if there exists a morphism $f^{-1}: B\rightarrow A$ such that $f\circ f^{-1} = \id_B$ and $f^{-1}\circ f = \id_A$.
\end{defn}
\begin{rem}
	The results shown about monic and epic morphisms imply that any isomorphism is monic and epic. However, the converse is not true as witnessed by the inclusion morphism $i$ described in the example above. If there exists an isomorphism between two objects $A$ and $B$, then they are \textbf{isomorphic}, denoted $A \cong B$. Isomorphic objects are also isomorphic in the opposite category as is easy to verify.
\end{rem}
\begin{defn}[Initial object]
	Let $C$ be a category, an object $A \in C_0$ is said to be \textbf{initial} if for any $B \in C_0$, $|\Hom_C(A,B)| = 1$, namely there are no two parallel morphisms with source $A$ and every object has a morphism coming from $A$.
\end{defn}
\begin{defn}[Terminal object]
	Let $C$ be a category, an object $A \in C_0$ is said to be \textbf{terminal} (or \textbf{final}) if for any $B \in C_0$, $|\Hom_C(B,A)| = 1$, namely there are no two parallel morphisms with target $A$ and every object has a morphism going to $A$.
\end{defn}
It is clear that an object is initial in a category $C$ if and only if it is terminal in $\op{C}$. Also, if an object is initial and terminal, we say it is a \textbf{zero} object and usually denote it $0$.
\begin{exmps}
	Here are examples of categories where initial and terminal objects may or may not exist.
	\begin{enumerate}
		\item $\exists$ terminal, $\nexists$ initial: Consider the poset $(\N, \geq)$ represented by the diagram below. It is clear that $0$ is terminal and no element can be initial because $0 \geq x$ implies $x = 0$.
		\begin{figure}[h]
			\centering
			\begin{tikzcd}
				\stackrel{0}{\bullet}  & \arrow[l] \stackrel{1}{\bullet}  & \arrow[l] \stackrel{2}{\bullet}  & \arrow[l] \cdots
			\end{tikzcd}
		\end{figure}
		\item  $\nexists$ terminal, $\exists$ initial: The category \textbf{GrpInj} where the objects are groups and the morphisms are injective homomorphisms only contains an initial object $\{1\}$. Indeed, an injective homomorphism $G \rightarrow H$ can be seen as subgroup of $H$ isomorphic to $G$. The identity group $\{1\}$ can only be isomorphic to the the identity subgroup as any other element has degree more than 1, so $\{1\}$ is initial. Moreover, any group $G$ cannot be terminal as $G \times (\Z/2\Z)$ cannot be isomorphic to any subgroup of $G$.
		\item $\nexists$ terminal, $\nexists$ initial: Let $G$ be a non trivial group. The category $G*$ has a single object $*$ with $\Hom_{G*}(*, *) = G$ and the composition rule being the multiplication in $G$. The only object $*$ cannot be initial nor terminal as $|\Hom_{G*}(*,*)| > 1$.
		
		The category whose objects are fields and morphisms are field homomorphisms also has no initial nor terminal objects because there are no field homomorphisms between fields of different characteristics.
		\item $\exists$ terminal, $\exists$ initial: Let $X$ be a topological space where $\tau$ is the collection of open sets (recall that it must contain $\emptyset$ and $X$). Let $T_X$ be the category representing the poset of open sets with inclusion as the relation. Namely, the objects are the open sets and for any two open sets $U, V \in \tau$, 
		\[\Hom_{T_X}(U,V) = \begin{cases}i_{U,V} & U \subseteq V\\ \emptyset & U \not\subseteq V\end{cases}\]
		Since the empty set is contained in every open set, it is an initial object. Since the full set $X$ contains every open set, it is a terminal object. No other set can be initial as it cannot be contained in $\emptyset$ nor be terminal as it cannot contain $X$. Moreover, note that the two objects are  not isomorphic as $\Hom_{T_X}(X, \emptyset) = \emptyset$.
	\end{enumerate}
\end{exmps}
\begin{prop}
	Let $C$ be a category and $A,B\in C_0$ be initial, then $A \cong B$.
\end{prop}
\begin{proof}
	Let $f$ be the single element in $\Hom_C(A,B)$ and $f'$ be the single element in $\Hom_C(B,A)$. Since the identity morphisms are the only elements of $\Hom_C(A,A)$ and $\Hom_C(B,B)$, $f' \circ f$ and $f\circ f'$, belonging to these sets, must be the identities. In other words $f$ and $f'$ are inverses, thus $A \cong B$. 
\end{proof}
The dual result follows.
\begin{prop}
	Let $C$ be a category and $A,B \in C_0$ be terminal, then $A \cong B$.
\end{prop}
\end{document}

