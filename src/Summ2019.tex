\PassOptionsToPackage{table}{xcolor}
\documentclass{beamer} %
\usefonttheme[onlymath]{serif}
\usepackage{mathpazo}  
%
% Choose how your presentation looks.
%
% For more themes, color themes and font themes, see:
% http://deic.uab.es/~iblanes/beamer_gallery/index_by_theme.html
%

\mode<presentation>
{
	\usetheme{default}      % or try Darmstadt, Madrid, Warsaw, ...
	\usecolortheme{beaver} % or try albatross, beaver, crane, ...
	\usefonttheme{default}  % or try serif, structurebold, ...
	\setbeamertemplate{navigation symbols}{}
	\setbeamertemplate{caption}[numbered]
} 

\usepackage[french]{babel}
\usepackage[utf8x]{inputenc}
\usepackage{tikz}
\usepackage{graphicx}
\usepackage{pgfplots}
\usepackage{multirow}
\title[P vs. NP]{How to prove \textbf{P} = \textbf{NP}?}
\author{Ralph Sarkis}
\institute{SUMM - 2019}
\date{11 Janvier 2019}
%--------Theorem Environments--------
%theoremstyle{plain} --- default
\newtheorem{thm}{Theorem}
\newtheorem{cor}[thm]{Corollary}
\newtheorem{prop}[thm]{Proposition}
\newtheorem{lem}[thm]{Lemma}
\newtheorem{conj}[thm]{Conjecture}
\newtheorem{quest}[thm]{Question}

\theoremstyle{definition}
\newtheorem{defn}[thm]{Definition}
\newtheorem{defns}[thm]{Definitions}
\newtheorem{con}[thm]{Construction}
\newtheorem{exmp}[thm]{Example}
\newtheorem{exmps}[thm]{Examples}
\newtheorem{notn}[thm]{Notation}
\newtheorem{notns}[thm]{Notations}
\newtheorem{addm}[thm]{Addendum}
\newtheorem{exer}[thm]{Exercise}

\theoremstyle{remark}
\newtheorem{rem}[thm]{Remark}
\newtheorem{rems}[thm]{Remarks}
\newtheorem{warn}[thm]{Warning}
\newtheorem{sch}[thm]{Scholium}
\begin{document}
	
	\begin{frame}
	\titlepage
\end{frame}

% Uncomment these lines for an automatically generated outline.
%\begin{frame}{Outline}
%  \tableofcontents
%\end{frame}

\section{Introduction}

\begin{frame}{Introduction}
\begin{block}{What are we talking about?}
	\pause
	\begin{itemize}
		\item Most famous open problem in theoretical computer science.
		\item Roughly translates to: If the solution to a problem can be verified easily, then can it be found easily?
		\item Has way too many theoretical and practical applications.
	\end{itemize}
\end{block}
\end{frame}

\begin{frame}{Outline}
	\begin{itemize}
		\pause
		\item Formal definitions
		
		\small{\textit{Computational models, Turing machines and complexity classes.}}
		\vskip 0.5cm
		\pause
		\normalsize
		\item Upper bounds: algorithms and reductions.
		
		\small{\textit{NP-hardness, Cook-Levin and examples.}}
		\vskip 0.5cm
		\pause
		\normalsize
		\item Lower bounds: three barriers.
		
		\small{\textit{Relativizing proofs, natural proofs and algebrizing proofs.}}
	\end{itemize}
\end{frame}
\begin{frame}{Origins of the question}
	\pause
	\begin{block}{Hilbert's 10th problem}
		Given a polynomial with integer coefficients in several variables: Find an algorithm that will determine whether it has integer roots or not.
	\end{block}
	
	\pause
	Emil Leon Post thought this was unsolvable and developed computability theory.
\end{frame}
\begin{frame}{Models of computation}
	Rigorous definition of an algorithm, often with notions of resources and efficiency.
	\pause
	\begin{block}{Examples}
		Turing machines, $\lambda$-calculus, finite state machines, RAM-model (modern computers), etc...
	\end{block}
\end{frame}
\begin{frame}{Turing machine}
	Defined as a 5-tuple $(Q, \Sigma, \Gamma, \delta, q_0)$ where
	\begin{enumerate}
		\item $Q$ is a set of states with distinct accept and reject states,
		\item $\Sigma$ (a finite set of symbols) is the input alphabet,
		\item $\Gamma \supseteq \Sigma$ (a finite set of symbols) is the tape alphabet,
		\item $\delta: Q \times \Gamma \rightarrow Q \times \Gamma \times \{L, R\}$ is the transition function, and
		\item $q_0$ is the initial state.
	\end{enumerate}
\end{frame}
\begin{frame}{Turing machine}
	\begin{block}{Running a TM}
		\begin{enumerate}
			\pause
			\item Start with input $w \in \Sigma^*$ on tape, head on first symbol and at state $q_0$.
			\pause
			\item At each step, follow the transition rule to change state, write on tape and move left or right.
			\pause
			\item Run previous step until accept or reject state is reached.
		\end{enumerate}
	\end{block}
	\pause
	A TM computes a function $f:\Sigma^* \rightarrow \Sigma^*$ if for any input $w$, it halts with $f(w)$ written on the tape.
\end{frame}
\begin{frame}{Resources of a TM}
	\begin{block}{Time}
		We say that $M$ runs in time $T(n)$ if for any input of size $n$, the number of steps that $M$ needs before halting is at most $T(n)$.
	\end{block}
	\pause
	\begin{block}{Space}
		We say that $M$ runs in space $S(n)$ if for any input of size $n$, the maximum number of cells of the tape used is at most $S(n)$.
	\end{block}
\end{frame}
\begin{frame}{Complexity classes}
	A complexity class is a set of functions $f: \Sigma^* \rightarrow \Sigma^*$ with \textquotedblleft similar\textquotedblright needs in terms of resources.
	\pause
	\begin{block}{Computable}
		A function $f$ is computable if there exists a TM that computes $f$ (no restriction on the resources).
	\end{block}
	\pause
	\begin{block}{\textbf{P}}
		A function $f$ is in \textbf{P} if there exists a TM that computes $f$ in time $T(n)$ for some polynomial $T$.
	\end{block}
	\pause
	\begin{block}{\textbf{NP}}
		A function $f$ is in \textbf{NP} if there exists a TM that, given $w,x \in \Sigma^*$, can say whether $f(x) = w$ in time $T(n)$ for some polynomial $T$.
	\end{block}
\end{frame}
\begin{frame}{Complexity classes}
	Complexity zoo.
\end{frame}


\begin{frame}{Examples}
	\begin{block}{Multiplication}
		Given two numbers $a, b \in \mathbb{N}$, output $a\cdot b$.
	\end{block}
	\pause
	\begin{block}{Factorization}
		Given a number $x \in \mathbb{N}$, output primes $p_1, \dots, p_n$ such that $p_1 \cdots p_n = n$.
	\end{block}
	\pause
	\begin{block}{Sudoku}
		Given a Sudoku board, output the solved board.
	\end{block}
	\pause
	\begin{block}{Bitcoin mining}
		Given the data for a block, output a nonce that satisfies the mining requirement.
	\end{block}
\end{frame}
\begin{frame}{Reductions}
	We say that a problem $A$ reduces to a problem $B$ if an algorithm that solves $B$ can be used to define an algorithm that solves $A$.
	
	
	\vspace{10pt}
	\pause
	We write $A\leq B$ because $B$ is at least as hard as $A$.
	
	\vspace{10pt}
	\pause
	When interested in polytime reductions, we write $A \leq_{p} B$.
\end{frame}
\begin{frame}{Cook-Levin}
	\begin{block}{Satisfiability (SAT)}
		Given a boolean formula $\phi$ in $n$ variables, decide whether there are values of $x_1, \dots, x_n$ such that $\phi(x_1, \dots, x_n)$.
	\end{block}
	\pause
	\begin{block}{Theorem}
		Any problem in \textbf{NP} has a polytime reduction to SAT.
	\end{block}
	\vspace{10pt}
	\pause
	Thus, if there exists an polytime algorithm for SAT, then $\textbf{P} = \textbf{NP}$.
\end{frame}
\begin{frame}{\textbf{NP}-hardness}
	\begin{block}{}
		If SAT reduces to a problem $X$, then $X$ is also \textbf{NP}-hard, hence solving $X$ in polytime implies $\textbf{P} = \textbf{NP}$.
	\end{block}
	\pause
	\begin{block}{Examples}
		Traveling salesperson problem, solving an $n\times n\times n$ Rubik's cube optimally, finding a valid move in an $n\times n$ checkers board.
	\end{block}
\end{frame}
\begin{frame}{Diagonalization}
	\begin{block}{Some functions are not computable}
		The number of TM is countable because each TM has a finite description $(Q, \Sigma, \Gamma, \delta, q_0)$.
		
		\pause \vspace{10pt}
		The number of languages is uncountable (cardinality of $\mathcal{P}(\Sigma^*)$).
	\end{block}
	\pause
	\vspace{20pt}
	Unfortunately, this kind of proof relativizes.
\end{frame}
\begin{frame}{Oracles}
	An oracle is a black box that can solve a particular problem in a single step.
	
	\pause
	\vspace{10pt}
	We write $\textbf{P}^L$ to denote the set of languages that can be decided in polytime with a TM that has access to an oracle for $L$.
	
	\pause
	\vspace{10pt}
	We say that a proof relativizes if it works whether or not oracles are allowed.
\end{frame}
\begin{frame}{Relativizing proofs}
	\begin{block}{\textbf{P} vs. \textbf{NP} has contradicting relativizations}
		\pause
		Baker, Gill and Solovay showed there are languages $A,B$ such that 
		\begin{align*}
			\textbf{P}^A &= \textbf{NP}^A\\
			\textbf{P}^B &\neq \textbf{NP}^B
		\end{align*}
	\end{block}
	\pause
	Thus, no relativizing proofs can lead to a solution.
\end{frame}
\begin{frame}{Circuits}
	\begin{block}{Definition}
		A directed acyclic graphs where each vertex is either an input with in-degree 0 or a gate computing AND, OR, or NOT of its inputs. One vertex is the output of the circuit. 
	\end{block}
	\pause
	\begin{block}{Resources}
		The size of a circuit is the number of edges and the depth of a circuit is the maximal length of path ending in the output gate.
	\end{block}
	\pause \vspace{10pt}
	\textbf{P/poly} denotes the set of languages that can be decided by a circuit of polynomial size.
\end{frame}
\begin{frame}{Natural proofs}
	\[\textbf{NP} \not\subseteq \textbf{P/poly} \implies \textbf{P} \neq \textbf{NP}\]
	
	\pause
	\begin{block}{Strategy}
		\pause
		\begin{enumerate}
			\item Find some property $X$ of functions that SAT (or some other \textbf{NP} problem) satisfies.
			\pause
			\item Show that no function in \textbf{P/poly} has property $X$.
		\end{enumerate}
	\end{block}
\end{frame}
\begin{frame}{Natural proofs}
	We say this is a natural proof if:
	\begin{enumerate}
		\pause
		\item \textquotedblleft Many\textquotedblright functions satisfy $X$. (Largeness)
		\pause
		\item Easy to verify if a function satisfies $X$. (Constructivity)
	\end{enumerate}
	
	\pause \vspace{10pt}
	Razbarov and Rudich showed that natural proof will inevitably fail.
\end{frame}
\begin{frame}{Algebrization}
	Similar to relativization, but we allow low-degree extension of oracles in finite fields.
\end{frame}
\begin{frame}{Reasons to believe $\textbf{P} = \textbf{NP}$}
	\begin{itemize}
		\pause
		\item All these barriers to prove $\textbf{P} \neq \textbf{NP}$.
		\pause
		\item We might find an $O(n^d)$ algorithm with $d$ huge.
		\pause
		\item We could prove existence of such an algorithm.
		\pause
		\item Upper bounders have been way more successful than lower bounders.
	\end{itemize}
\end{frame}
\begin{frame}{Reasons to believe $\textbf{P} \neq \textbf{NP}$}
	\begin{itemize}
		\pause
		\item No one has been able to solve \textbf{NP-complete} problems in polytime.
		\pause
		\item If $\textbf{P} = \textbf{NP}$, then $\textbf{P} = \textbf{PH}$ and \textbf{PH} includes so many harder problems (e.g.: proving theorems).
	\end{itemize}
\end{frame}

\begin{frame}{Thank you !}
	Merci !
\end{frame}
\end{document}
